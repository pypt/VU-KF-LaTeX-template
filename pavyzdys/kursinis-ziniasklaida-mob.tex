\documentclass[kursinis-darbas]{vukf}

\begin{document}
	
\title{Žiniasklaidos kanalai mobiliuosiuose įrenginiuose}
\vukfTitleEn{Mass media channels in mobile devices}
\author{Linas Valiukas}
\vukfAuthorFirstName{Linas}
\vukfAuthorLastName{Valiukas}
\vukfAuthorDescription{Žurnalistikos bakalauro studijų programos \\ II kurso II grupės studentas}
\vukfDivision{Žurnalistikos institutas}
\vukfDivisionEn{Institute of journalism}
\vukfKeywords{žiniasklaida, informacinės technologijos, mobilieji telefonai, Apple iPhone, Java, prekės ženklai, Adobe Flash, Ustream, piliečių žurnalistika}
\vukfKeywordsEn{mass media, information technology, mobile phones, Apple iPhone, Java, brand names, Adobe Flash, Ustream, civic journalism}
\vukfSupervisor{lekt. Džiugas Paršonis}
% FIXME 2010 metai

\maketitle
\tableofcontents


\begin{vukfAbstract}
	
Šis kursinis darbas analizuoja dvipusius mainus tarp žiniasklaidos ir jos vartotojų mobiliaisiais kanalais.

Pirmojoje dalyje analizuojama, kokiais būdais žiniasklaida pasirinktu mobiliuoju kanalu (ar keliais kanalais) teikia turinį savo vartotojams. Populiariausios praktikos šiai mainų pusei realizuoti yra specialaus mobiliojo tinklalapio sukūrimas, įdiegiamos programinės įrangos sukūrimas arba šių dviejų priemonių derinimas tarpusavyje. Darbas apibūdina įvairias metodologijas, naudojamas pasirenkant kurią nors iš šių priemonių. Taip pat yra palyginami populiariausi Lietuvos naujienų portalai – nustatoma, kokias mobiliąsias priemones jie naudoja tam, kad pasiektų savo auditoriją.

Antrojoje dalyje analizuojama, kaip mobiliųjų technologijų pagalba realizuojama piliečių žurnalistika. Apibūdinami žiniasklaidos vartotojų motyvai, metodai kurti ir teikti turinį žiniasklaidai.

\end{vukfAbstract}

\begin{vukfAbstractEn}

This study focuses on the two–way exchange between mass media and its consumers via the mobile channels.

One of those ways is that the mass media provides content to the consumers using some particular mobile channel. Common practices for implementing that channel is creating a mobile–adopted website for accessing the mass media outlet's content, creating a designated installable mobile application for doing that, or both. Study discusses various methodologies used for each of these approaches. Also, most popular online news outlets in Lithuania are compared in terms of how they use mobile channels for reaching their readership.

The second part of this study analyzes a part of civic journalism in terms of using mobile devices for creating content for it. Motives, methods and specific cases for creating such type of journalism are discussed.

\end{vukfAbstractEn}


%
% Glosarijus
%

%\newacronym{FCC}{FCC}{\emph{Federal Communications Commision} („Federalinė komunikacijų komisija“, nacionalinis JAV transliacijų reguliuotojas)}

%\newacronym{Ofcom}{„Ofcom“}{\emph{Office of Communications} (nacionalinis Didžiosios Britanijos transliacijų ir telekomunikacijų reguliuotojas)}

%\newacronym{BBC}{BBC}{\emph{British Broadcasting Corporation} (nacionalinis Didžiosios Britanijos radijo ir televizijos transliuotojas)}


%
% Įvadas
%

\vukfIntroduction

% **Įvade** pristatoma darbo tema, aptariamas ir pagrindžiamas darbo aktualumas, nurodomas kursinio darbo tikslas ir uždaviniai, pristatomi darbe panaudoti tyrimo metodai, apibendrintai aptariama literatūra, atskleidžiama darbo struktūra. Rekomenduojama įvado apimtis – 5–10 proc. kursinio darbo apimties.

Šiame kursiniame darbe tiriami ir analizuojami dvikrypčiai mainai tarp žiniasklaidos ir jos vartotojo. Viena šių mainų kryptis yra ta, kad žiniasklaida pastaruoju metu pradeda įžvelgti mobiliųjų įrenginių rinkos potencialą ir bando į ją skverbtis, pateikdama įvairius technologinius sprendimus, kurių pagalba vartotojas gali lengviau ir paprasčiau vartoti konkretaus žiniasklaidos šaltinio sukuriamą turinį. Šie mainai yra dvikrypčiai, nes ir pats vartotojas, dažnai paskatintas pačios žiniasklaidos, naudodamasis mobiliaisiais įrenginiais kuria ir teikia turinį žiniasklaidai – praneša apie įvykius, naujienas, siūlo temas; fotografuoja, įrašo garsą, filmuoja, sukurtą medžiagą siųsdamas žiniasklaidos vertinimui.

Ši tema yra aktuali ir nagrinėtina ne tik dėl didelės, gal net milžiniškos mobiliųjų technologijų plėtros ir skvarbos (pasak \emph{Gartner, Inc.} atlikto tyrimo išvadų \cite{gartner_highlights_key_predictions_for_it_organisations_and_users_in_2010_and_beyond}, iki 2013 m. mobilieji telefonai taps populiariausia žiniatinklio naršymo priemone, taip aplenkdami personalinius kompiuterius). Technologijų, kurias jau kuria arba dar tik ruošiasi kurti žiniasklaida, rinkoje yra svarbių technologijos specifikos, tarpininkaujančių įmonių požiūrio ir kitų aspektų, ties kuriais galima nesunkiai „suklupti“, padaryti realizacinę klaidą. Vis dėlto, mobilioji rinka yra ta rinka, kurios žiniasklaida dar gerai iki galo neatrado ir neišnaudojo; joje yra vartotojų, taigi – ir potencialių pajamų žiniasklaidos bendrovėms. Šio „papildomo uždarbio“ galimybė ir geriausios to panaudojimo praktikos yra dar vienas šio kursinio darbo aktualumo aspektas.\\

\textbf{Kursinio darbo tikslas}: apžvelgti žiniasklaidos ir vartotojo turinio mainus, vykstančius mobiliaisiais kanalais.\\

\textbf{Kursinio darbo uždaviniai}:

\begin{itemize}
	\item nurodyti ir apibūdinti egzistuojančias praktikas, kaip žiniasklaida teikia turinį vartotojui, jas palyginti;
	\item išskirti vadinamuosius „mobilumo lygius“ pagal tai, kokius ir kiek sprendimų žiniasklaida naudoja sklaidai mobiliaisiais kanalais;
	\item įvertinti Lietuvos žiniasklaidos bendrovių įnašą į mobiliąją rinką ir nustatyti jų naudojamus mobiliuosius sprendimus;
	\item apibūdinti žiniasklaidos vartotojų norą teikti informaciją pačiai žiniasklaidai; nurodyti tokio elgesio atvejus ir motyvus.
\end{itemize}

Kadangi technologijos, o ypač – mobiliosios technologijos, labai dideliu greičiu keičiasi ir tobulėja, taip vis keisdamos situaciją žiniasklaidos mobiliųjų kanalų edvėje, akademinių šaltinių apie analizuojamą objektą nėra daug, taigi šiame darbe be jų kartu remtasi ir pačios žiniasklaidos pranešimais, vertinimais, taip pat remiantis statistika ir kitais faktiniais duomenimis darytos naujos, galbūt niekur kitur neskelbtos išvados.

Kursinis darbas susideda iš didesniosios dalies, apibūdinančios žiniasklaidą, teikiančią turinį vartotojui mobiliaisiais kanalais, ir mažesniosios, apibūdinančios vartotojus, teikiančius turinį žiniasklaidai.


\section{Žiniasklaida mobiliaisiais kanalais teikia turinį vartotojui}

\subsection{Žiniasklaidos turinio pateikimo vartotojui mobiliaisiais kanalais tipai}

Nors mobilusis įrenginys, pavyzdžiui, mobilusis telefonas, yra daugialypis ir daugiafunkcinis kūrinys, žiniasklaidos bendrovės, norėdamos pasiekti auditoriją, besinaudojančią mobiliaisiais įrenginiais, paprastai nueina vienu arba abiejais iš tik dviejų kelių: sukuria bei publikuoja žiniatinklio (angl. \emph{World Wide Web}) tinklalapį, skirtą ir pritaikytą mobiliesiems įrenginiams, ir/arba sukuria bei publikuoja specialią įdiegiamą programinę įrangą (toliau – PĮ), pritaikytą kuriai nors mobiliųjų telefonų naudojamai operacinei sistemai.

\subsubsection{Mobiliesiems įrenginiams pritaikytas žiniatinklio tinklalapis}

Vienas iš „kelių“, kuriuo žiniasklaidos bendrovė (ar bet kuri kita, kokį nors vartotojui potencialiai įdomų ir/arba aktualų turinį kurianti, organizacija) gali „nueiti“, yra specialaus, mobiliojo įrenginio specifikai pritaikyto tinklalapio sukūrimas ir palaikymas.

Nuo būtent šio metodo savo kuriamo turinio publikavimui, išbandydama mobiliąją erdvę, pradeda dauguma žiniasklaidos bendrovių – viena vertus, dėl to, kad šis metodas atsidaro ir paplito žymiai anksčiau už vėliau aprašomą „įdiegiamos programos“ metodą; taip pat šis metodas nereikalauja daug papildomų tyrimų ir išsimokslinimo (angl. \emph{learning curve}), taigi jis gali būti pakankamai greitai realizuojamas ir išbandomas.

Technologija mobiliajam tinklalapiui kurti pasirenkama pagal jo tikslinę auditoriją, kuri vertinama pagal tai, kokio „naujumo“ mobiliaisiais įrenginiais bus atveriamas kuriamasis tinklalapis, bei pagal tai, kokias funkcijas privalės vykdyti kuriamasis tinklalapis.

Jeigu tikslinė auditorija naudoja senesnio modelio mobiliuosius įrenginius (pavyzdžiui, telefonus), o funkcionalumas apsiriboja, pavyzdžiui, straipsnių atvėrimu ir skaitymu, galima rinktis WML (angl. \emph{Wireless Markup Language}) žymėjimo kalbą, kuri yra labiau suderinama su senesio modelio įrenginiais, greičiau juose interpretuojama, bet turi funkcionalumo trūkumų, o taip pat yra išvis nebepalaikoma naujesnio modelio mobiliuosiuose įrenginiuose, pavyzdžiui, \emph{Apple iPhone}.

Jeigu numatoma, kad tinklalapyje turėtų lankytis sąlyginai naujesnių įrenginių savininkai, žiniasklaidos bendrovė galėtų pasirinkti naujesnius, daugiau funkcijų turinčius XHTML (angl. \emph{eXtensible HyperText Markup Language}) arba XHTML Mobile žymėjimo kalbų (technologijų) variantus. Pastarasis variantas bendrovei galėtų pasirodyti patrauklus ir lengvai realizuojamas ir dėl to, kad šis nedaug skiriasi nuo technologijų, naudojamų staliniams kompiuteriams skirtų tinklalapių kūrimui, todėl, priklausomai nuo funkcionalumo, gali užtekti tiesiog papildyti jau veikiantį staliniam kompiuteriui skirtą tinklalapį (atlikti pritaikymo darbus), ir šis gerai arba bent jau patenkinamai veiks ir mobiliuosiuose įrenginiuose.

Mobilusis tinklalapis pasižymi dauguma šių savybių:

\paragraph{Naudojamos technologijos, panašios į tas, kurios jau įsisavintos}

Mobilusis tinklalapis naudoja kiek kitas (bet panašiai realizuojamas ir veikiančias) žymėjimo ir programavimo kalbų technologijas nei tinklalapiai paprastiems personaliniams kompiuteriams, arba naudoja lygiai tas pačias technologijas, bet „aštriau“ laikosi pasirinktos žymėjimo kalbos standartų. Pavyzdžiui, naujienų portalas savo pagrindiniam tinklalapiui publikuoti naudoja kiek nuo standartų „nukrypstantį“ XHTML, o tinklalapiui, skirtam mobiliesiems įrenginiams, naudoja tiksliai standartus atitinkantį XHTML. Tikslus standartizacijos palaikymas reikalauja papildomų žinių ir investicijų.

\paragraph{Mobilusis tinklalapis – labiau ribotas}

Mobilusis tinklalapis yra labiau ribotas nei įprastas tinklalapis. Mobilieji telefonai, kurių pagalba vartojamas mobiliojo tinklalapio turinys, nepalaiko įvairių technologijų, kurios suteikia galimybę tinklalapio kūrėjams naudoti tam tikrus vaizdo, garso, interakcijos elementus (pavyzdžiui, \emph{Adobe Flash}), arba šias technologijas palaiko nepilnai. Dėl to mobilusis tinklalapis, lyginant su „įprastu“ tinklalapiu, dažnai atrodo kaip „apkarpyta“ pastarojo kopija – be interaktyvių apklausų, išvaizdžios nuotraukų, video medžiagos peržiūros galimybės. Šiuo metu skirtingų gamintojų gaminamuose mobiliuosiuose įrenginiuose įdiegtos daugmaž tos pačios technologijos, bet jų realizacijos yra skirtingos, todėl norint sukurti pilnavertį produktą mobiliajai rinkai, žiniasklaidos bendrovei tektų skirti daug laiko, pastangų ir bandymų; dažnai – be aiškios perspektyvos ir vartotojų masės (nes, pavyzdžiui, kuri nors konkreti technologinė priemonė yra realizuota ir naudojama tik vienam kuriam nors nišiniam telefono modeliui).

\paragraph{Nedidelis atsiunčiamų duomenų dydis baitais}
\label{sec:nedidelis_atsiunciamu_duomenu_dydis_baitais}

Mobilusis tinklalapis yra mažesnis atsiunčiamų duomenų dydžio prasme. Mobilieji telefonai prie interneto paprastai būna prijungti sąlyginai (lyginant su personalinių kompiuterių vidurkiu) lėtu ir brangiu interneto ryšiu. Taip pat mobilusis įrenginys beveik visada turi lėtesnį procesorių, mažesnį laikinosios atminties dydį ir kitus parametrus, kurie yra menkesni už personalinio kompiuterio. Todėl tinklalapiai, skirti mobiliesiems įrenginiams, bent jau kol kas privalo būti nedideli (dydžiu baitais), kad mobilusis įrenginys gebėtų juos atsisiųsti, apdoroti ir pateikti vartotojui ir taip pat „sutilpti“ į savo turimus resursus.

Tai yra dar vienas iššūkis žiniasklaidos bendrovės samdomiems tinklalapių kūrėjams, o kartu ir pačiai bendrovei – kūrėjams reikia susipažinti su įvairių telefonų modelių teikiamomis galimybėmis ir egzistuojančiomis skaitinėmis ribomis, žiniasklaidos bendrovei – nuspręsti, kuris jos kuriamas turinys yra pats svarbiausias ir/arba įdomiausias mobiliajam vartotojui, taigi kurį publikuoti mobiliajame tinklalapyje. Tai taip pat reikalauja papildomo darbo ir investicijų.

\paragraph{Mažesnis žiniasklaidos turinio kiekis}

Mobilusis tinklalapis yra mažesnis pateikiamos informacijos prasme. Šis faktorius „išplaukia“ iš aukščiau paminėtojo faktoriaus \ref{sec:nedidelis_atsiunciamu_duomenu_dydis_baitais} ir iš dalies yra jo pasekmė.

Pirmiausia, turinio kiekis mobiliesiems įrenginiams pritaikytuose tinklalapiuose paprastai būna mažesnis dėl to, kad didesnis jo kiekis reikštų didesnį puslapį, todėl ir didesnį atsiunčiamų duomenų kiekį – lėtesnį tinklalapio atvertimo greitį, ilgesnį vieno puslapio interpretavimo greitį mobiliojo įrenginio pusėje.

Kitas faktorius yra tai, kad mobiliojo telefono interfeisas yra mažiau patogus už personalinio kompiuterio interfeisą – pastarasis turi klaviatūrą su daugiau nei 100 klavišų, pelės įrenginį, su kuria galima „paspausti“ ant dominančių objektų ekrane milimetro dalies tikslumu, taip pat – daug didesnį už įprasto mobiliojo įrenginio ekraną. Tuo tarpu dažnas mobilusis įrenginys, pavyzdžiui, mobilusis telefonas, turi tik riboto pločio ir aukščio ekraną su nedideliu spalvų skaičiumi, kurį šis gali atvaizduoti, bei skaičių klaviatūrą ir prie jos pridedamus kelis funkcinius klavišus (apie 11 – 14 klavišų, jeigu lygintume su 100 ir daugiau klavišų turinčiu personaliniu kompiuteriu). Vartotojas yra priverstas navigaciją po visą tinklalapį ar po vieną konkretaus tinklalapio puslapį atlikti šiomis ribotomis galimybėmis, o kuo daugiau puslapis turi turinio (galimų pasirinkimų, kur vartotojas gali „nueiti“ ar „paspausti“), tuo sunkiau jis (vartotojas) tai gali padaryti, taigi žiniasklaidos kūrėjams tenka riboti mobiliesiems įrenginiams pateikiamo turinio apimtis vien dėl to, kad pateikti didelius kiekius šio turinio tiesiog būtų neparanku pačiam vartotojui.

Trečiasis faktorius yra susijęs su tuo, kur, kada ir kokiomis aplinkybėmis mobiliųjų telefonų naudotojai yra linkę vartoti žiniasklaidą (skaityti, klausytis, žiūrėti žiniasklaidos produktus). Mobilusis įrenginys, prisimenant jo pavadinimą, yra mobilus, taigi reiktų tikėtis, kad itin dažnai šie yra naudojami neramioje, judrioje, nepastovioje, turinčioje daug trikdžių aplinkoje – viešajame transporte, automobilyje, menkiau sudominusioje paskaitoje ir kt. Skiriasi įvairios aplinkybės, kurios daro naudojimąsi mobiliąja žiniasklaida kitokiu, nei įprastąja – telefono ekrane esantis turinys gali būti bandomas įskaityti esant per tamsiam arba per šviesiam apšvietimui, triukšme, netgi sąlyginai pavojingoje situacijoje (pavyzdžiui, einant per judrios gatvės perėją). Todėl turinys mobiliajame telefone negali būti ilgas, reikalaujantis įsigilinimo, įsidėmėjimo, įžvalgos, pamąstymų – tiesiog būdas, kuriuo jis paprastai suvartojamas, to dažnai neleidžia.


\subsubsection{Įdiegiama programinė įranga mobiliajam įrenginiui}

\paragraph{„Java Mobile“ modelis}

Iki Apple iPhone mobiliojo įrenginio pasirodymo taip egzistavo būdai turiniui mobiliesiems telefonams pateikti ir apmokestinti. Daugumoje šiuo metu veikiančių įprastų – nepriskirtinų „išmaniojo telefono“ (angl. \emph{smartphone}) klasei – telefonų, yra įdiegta galimybė vykdyti „Java Mobile“ (arba J2ME) technologija sukurtą programinę įrangą \cite{java_com_learn_about_java_technology}. Reiktų paminėti, kad telefone iPhone ši technologija neįdiegta ir neveikia \cite{apple_com_iphone_technical_specifications}.

Nors paprastai telefonų, kuriuose veikia „Java Mobile“, techninės specifikacijos būna menkiau pritaikytos žiniasklaidos ar kitų medijos kanalų turinio vartojimui (dėl mažesnio telefono ekrano, prastesnės skiriamosios gebos, lėtesnio telefone įmontuoto procesoriaus greičio ir kt.), galimybė kažką sukurti ir paleisti egzistavo (ir vis dar egzistuoja) jau kurį laiką – nuo „Java Mobile“ platformos atsidarimo 1999 m. \cite{sun_com_j2me_technology_turns_5}

„Java Mobile“ neišplitimą ir sąlyginį nepopuliarumą tarp žiniasklaidos bendrovių galima būtų priskirti šiems faktoriams:

\begin{itemize}
	\item Nebuvo vienos (ir vienintelės) mobiliųjų telefonų PĮ parduotuvės, kurią vienaip ar kitaip populiarintų visi mobiliųjų telefonų, palaikančių „Java Mobile“ technologiją, gamintojai.
	\item Buvo (ir vis dar yra) sudėtinga vieną konkrečią programą pritaikyti visiems ar bent jau daugumai „Java Mobile“ technologiją palaikančių telefonų dėl jų specifikacijų skirtumų. Vienas esminių skirtumų tarp „Java Mobile“ telefonų yra ekrano dydis, kuris skiriasi tarp atskirų gamintojų gaminamų telefonų ir telefonų modelių, kuriuos gamina vienas ir tas pats gamintojas. Taip pat skiriasi šių telefonų klaviatūros (valdymo, skaičių klaviatūrų išdėstymai), palaikomos techninės galimybės (mobiliojo telefono geografinės vietovės nustatymo galimybė, ekrano spalvų skaičius ir kt.) Pritaikyti vieną ir tą pačią programą daugeliui skirtingų telefonų yra sunkus ir daug kainuojantis uždavinys. \emph{Apple} metodika šiuo atveju laimi tuo, kad asmuo ar bendrovė, nusprendusi kurti ir išleisti PĮ mobiliajam telefonui \emph{iPhone}, gali būti tikra tuo, kad sukurtoji PĮ veiks lygiai taip pat gerai visuose telefono modeliuose.
	\item Mobiliųjų telefonų naudotojai buvo nepakankamai informuoti apie PĮ savo mobiliajam telefonui susiradimą, įsidiegimą, bandymą, naudojimą, ir buvo neužtektinai skatinami tai daryti. Norint išbandyti programinę įrangą, tekdavo (ir vis dar tenka) jos pačiam ieškotis internete, skirtingose „Java Mobile“ PĮ parduotuvėse. Susiradus, naudotojas turi mobiliajame telefone susirasti žiniatinklio naršyklę, mobiliojo telefono klaviatūra pats įvesti parduotuvės adresą internete, prieš siųsdamasis atsižvelgti į įvairius parametrus (pavyzdžiui, PĮ kūrėjo nurodomą rekomenduojamą ekrano dydį ir/arba telefono modelį korektiškam programos veikimui), ir vis tiek neretai parsisiųsta „Java Mobile“ programa veikia per lėtai, nekorektiškai, ribotai (pačiame telefone neturint reikiamos techninės įrangos tam tikroms funkcijoms atlikti pageidaujamu greičiu arba atlikti apskritai). Dažnas vartotojas iš galimybių, teikiamų „Java Mobile“ technologijos pagalba, nematė sau konkrečios panaudos, ir todėl „Java“ PĮ naudojimo opciją nurašydavo į užribį.
\end{itemize}

Šios aplinkybės susidėdavo į tai, kad bendrovės – ne tos, kurių pagrindinė verslo kryptis yra programinės įrangos kūrimas, o tos, kurios užsiima su mobiliųjų telefonų PĮ nesusijusia veikla ir kurioms žingsnis į mobilųjį pasaulį nėra gyvybiškai būtinas pasirinkimas (pavyzdžiui, žiniasklaidos bendrovės) – atsargiai žiūrėjo į „Java Mobile“ PĮ rinką.


\paragraph{iPhone modelis}

Idėja sukurti ne tik mobilujį tinklalapį žiniasklaidos turiniui pateikti, bet ir į patį mobilujį įrenginį įdiegiamą programą to turinio vartojimui, nebuvo populiari iki \emph{Apple iPhone} (toliau – \emph{iPhone}) mobiliojo įrenginio pristatymo. Su \emph{iPhone} išleidimu 2007 m. pradžioje \cite{apple_com_reinvents_the_phone_with_iphone} ir po daugiau nei metų sekusios vadinamosios \emph{iTunes App Store} (toliau – \emph{App Store}) programinės įrangos parduotuvės sukūrimu 2008 m. \cite{apple_com_introduces_the_new_iphone_3g}, iš naujo gimė taikomųjų programų mobiliesiems telefonams era.

\emph{iPhone} sėkmę žiniasklaidoje lėmė įvairios techninės, socialinės \cite{theregister_why_the_iphone_is_success_not_for_the_reasons_you_think} ir verslo aplinkybės.

Prie techninių aplinkybių būtų galima priskirti tai, kad šis telefonas–mobilusis įrenginys turėjo nepalyginamai didesnį ekraną negu prieš tai išleisti mobilieji telefonai ar kompiuteriai. Taip pat \emph{iPhone} buvo vienas pirmųjų mobiliųjų telefonų, valdomų prisilietimu, ir pirmasis tokio tipo telefonas, taip išpopuliarėjęs pasaulyje.

Pagrindinė verslo aplinkybė, lėmusi žiniasklaidos ir kitokio tipo PĮ telefone atgimimą, buvo ta, kad JAV bendrovė \emph{Apple, Inc.} (toliau – \emph{Apple}) išleido ne tik išvaizdų, novatorišką, patogų naudoti, „draugišką“ telefoną, bet kartu „po juo“ kiek vėliau „pastatė“ ir stiprią, gerai apgalvotą turinio leidimo platformą \cite{oreilly_iphones_app_stores_and_ecosystems}.

Didelę dalį telefono \emph{iPhone} sėkmės žiniasklaidos rinkoje (lyginant su „Java Mobile“) lėmė tai, kad sukurtas ir išleistas buvo ne tik pats mobilusis įrenginys, bet ir metodas (infrastruktūra, modelis) įvairiems kūrėjams ir kūrėjų kompanijoms (tame tarpe ir žiniasklaidos bendrovėms) kurti ir publikuoti turinį šiam mobiliajam įrenginiui.

\paragraph{Įdiegiamos programos privalumai}

\subparagraph{Platesnės techninės galimybės}

Programa, priešingai nei daugmaž izoliuotas nuo pačio įrenginio tinklalapis, geba panaudoti konkretaus mobiliojo įrenginio teikiamas technines galimybes \cite{quirksmode_native_iphone_apps_vs_web_apps}. Programos kūrėjas (pavyzdžiui, žiniasklaidos bendrovė) gali teikti savo vartotojams galimybę nufotografuoti pastebėto įvykio nuotraukas ir jas atsiųsti į redakciją, prie jos pridėti geolokacinę įvykio informaciją (įvykio ilgumos ir platumos koordinates, nustatytas pasinaudojus telefone įdiegta GPS technologija); taip pat išsisaugoti patikusį straipsnį ar nuotrauką mobiliajame telefone. Programos kūrėjas gauna priėjimą prie visos telefono ekrano erdvės (priešingai nei tinklalapio kūrėjas, kuris lieka „įspraustas“ telefono naršyklės ekrane) ir gali ją išnaudoti savo tikslams geriau pasiekti.

\subparagraph{Greitis}

Programa paprastai veikia greičiau, patikimiau ir „dailiau“ nei tinklalapis \cite{quirksmode_native_iphone_apps_vs_web_apps}. Tai daugiausia yra susiję su techniniais mobiliųjų telefonų veikimo faktoriais. Pavyzdžiui, specialiai konkretaus žiniasklaidos kanalo turiniui pritaikyta programa yra kuriama taip, kad siųstų tik pačius duomenis – straipsnius ir prie jų esančius duomenis, o naršyklė, atidarydama tinklalapį, privalo atsisiųsti krūvas papildomos informacijos – aprašymus, kaip mobilusis tinklalapis turi atrodyti, įvairias funkcijas apibūdinančius failus, iliustracijas, tarnaujančias vien tik tinklalapio išvaizdos palaikymui ir kt. Šiuo atveju, naudojant įdiegiamą programą, taupomas atsiunčiamų duomenų kiekis, taigi ir naudotojo laikas. Taip pat, kaip jau minėta, programa, skirtingai nei tinklalapis, tiesiogiai „prieina“ prie mobiliojo telefono aparatinės įrangos, o be tarpininko ji geba greičiau, aiškiau ir kokybiškiau piešti grafinius elementus.

\subparagraph{Prekės ženklo matomumas}

Įdiegiama programa arba jos ikona dėl telefono vartotojo interfeiso struktūros dažnai yra visą laiką rodoma telefono ekrane, todėl vartotojas dažniau apie ją prisimena \cite{imediaconnection_secrets_behind_3_branded_iphone_app_successes}. Vartotojui, norinčiam atsidaryti tinklalapį mobiliojo telefono naršyklėje, tenka nueiti ilgoką kelią: telefone susirasti naršyklės programą, naršyklės programoje įvesti žiniasklaidos publikuojamo žinių šaltinio žiniatinklio adresą (angl. \emph{URL}) arba išsirinkti šį adresą iš pažymėtųjų (angl. \emph{bookmarked}), tuomet laukti, kol pasirinktasis tinklalapis atsivers. Tuo tarpu atskiroms programoms mobiliuosiuose telefonuose paprastai yra suteikiama didesnė svarba ir prioritetas – šios gauna „teisę“ savo ikoną rodyti pagrindiniame telefono ekrane (pavyzdžiui, \emph{iPhone} atveju) arba jame net visą laiką aktyviai veikti (pavyzdžiui, \emph{Nokia N97} ir dalies kitų \emph{Nokia} gaminamų telefonų palaikomos \emph{widget} technologijos atveju \cite{nokia_whats_the_latest_about_web_runtime_from_nokia}). Vartotojas, norėdamas atverti žiniasklaidos mobilųjį tinklalapį, turi iš anksto sau būti tarsi susikūręs „užduotį“, „planą“ tai padaryti; jis turi būti itin aktyvus ir labai besidomintis žiniasklaidos kuriamu turiniu. Pati žiniasklaida mobiliojo telefono pagalba neturi galimybių vartotojui pati apie save priminti. Kitu, designuotos įdiegiamos programos sukūrimo atveju, ši programa (pavyzdžiui, jos ikona) yra daug dažniau matoma vartotojo, taigi šis dažniau pasileidžia pačią programą, taigi – dažniau vartoja žiniasklaidos bendrovės kuriamą turinį.


\paragraph{Įdiegiamos programos trūkumai}

\subparagraph{„Pririšimas“ prie vienos konkrečios platformos}

Didelis programos kūrimo trūkumas yra tas, kad ši paprastai veikia tik vieno mobiliųjų įrenginių arba operacinių sistemų jiems gamintojo kuriamuose produktuose. Pavyzdžiui, \emph{iOS} (arba \emph{iPhone OS}) operacinei sistemai sukurta programa veiks vien tik \emph{Apple iPhone} ir \emph{Apple iPad} įrenginiuose ir neveiks kituose, pavyzdžiui, \emph{Nokia} arba \emph{Siemens} gaminiuose. Tai verčia žiniasklaidos ir kitas bendroves, susidomėjusias programinės įrangos kūrimu mobiliesiems telefonams, rinktis, kuris (kurie) iš mobiliosios rinkos dalyvių yra paklausiausias, įdomiausias ar labiausiai atspindintis esamą arba siekiamą žiniasklaidos (arba kitos bendrovės) auditoriją. Programų kūrimo metodikos skirtingoms mobiliųjų įrenginių operacinėms sistemoms labai skiriasi, taigi sukurti, o svarbiausia – diena iš dienos palaikyti programinės įrangos veikimą daugiau nei vienai operacinei sistemai (platformai) gali būti brangu ir neefektyvu.

\subparagraph{Gamintojo taikomi ribojimai publikuojamam turiniui}

Mobiliųjų telefonų gamintojų tarpininkavimo tarp mobiliojo telefono vartotojo ir programinės įrangos gamintojo atveju, galimi tam tikri iš žiniasklaidos pusės nepageidaujami ribojimai, atstovaujantys mobiliojo telefono gamintojo interesus. Geras pavyzdys būtų bendrovė Apple, jos gaminamas populiarus išmanusis telefonas iPhone ir jam skirtų programų internetinė parduotuvė App Store. Norint siūlyti programinę įrangą telefonui iPhone, privaloma ją publikuoti App Store parduotuvėje, o ši naujai programinei įrangai taiko reikalavimus \cite{apple_iphone_developer_program_license_agreement}, ne visada suderinamus su įvairių žiniasklaidos bendrovių siūlomu turiniu, pavyzdžiui:

\begin{itemize}
	\item 2009 m. iš \emph{App Store} buvo pašalinta elektroninių knygų programa „Eucalyptus“ dėl to, kad šios pagalba buvo galima atsisiųsti antikinę erotinio turinio knygą \emph{Kama Sutra} \cite{guardian_apple_backtracks_on_iphone_sex_ban}. Per \emph{App Store} gyvavimo laiką, dalis žiniasklaidos bendrovių sukurtų programų buvo nepriimtos į parduotuvę, nes šios viešino kurios nors rūšies (tekstinės arba grafinės) erotinį turinį; riba, ties kuria \emph{Apple} vertindavo turinį kaip erotinio pobūdžio, nebuvo pastovi ir nuolat keitėsi. Tarp nepriimtųjų yra laikraščiai bei žurnalai The Sun \cite{guardian_the_suns_obscene_page_3_girls_get_iphone_newspaper_app_banned_by_apple}, Der Spiegel \cite{nytimes_publishers_question_apples_rejection_of_nudity}, Stern \cite{nytimes_publishers_question_apples_rejection_of_nudity}, Bild \cite{guardian_german_publisher_in_row_with_apple_over_pin_ups_in_iphone_app}.
	\item 2009 m. į \emph{App Store} parduotuvę buvo nepriimta Pulitzerio prizo laimėtojo Mark Fiore pašaipius komiksus publikuoti skirta programa. Oficiali priežastis nepriimti programos buvo ta, kad ši „šaiposi iš viešų asmenų“. Vėliau \emph{Apple} persigalvojo, ir šiuo metu programa jau publikuojama \emph{App Store} \cite{nytimes_a_pulitzer_winner_gets_apples_reconsideration}.
\end{itemize}

\emph{Apple} bendrovės sukurta ir naudojama \emph{App Store} parduotuvė yra vienintelis šaltinis, iš kurio \emph{iPhone} naudotojai gali atsisiųsti programinę įrangą. Nekeičiant pačio telefono operacinės sistemos, neįmanoma trečiajam asmeniui ar bendrovei pateikti PĮ šiam telefonui apeinant \emph{App Store}.

Šis \emph{Apple} sprendimas – „uždaryti“ telefoną, t.y. leisti įdiegti taikomąją PĮ į jį tik vienaip ar kitaip tarpininkaujant pačiai \emph{Apple}, susilaukia ir pagyrų, ir kritikos.

Viena vertus, įvairūs autoriai mano, kad toks metodas publikuoti PĮ buvo pasirinktas dėl to, kad \emph{Apple} monopoliškai kontroliuotų savo gaminamame įrengyje veikiančią taikomąją PĮ, taigi galėtų užtikrinti, kad į nė vieno iš klientų telefonų nepateks neišvaizdi, prastai veikianti, „lūžinėjanti“, vartotoją apgaudinėjanti ar kitokia nepageidautina programinė įranga. Esą, vartotojas pasąmonėje neatskiria pačio \emph{Apple} gaminamo telefono ir kitų gamintojų kuriamos PĮ šiam telefonui, taigi suponuotų \emph{iPhone} telefoną su prasta PĮ kokybe, ir todėl kristų pačio iPhone telefono įvaizdis. \emph{Apple}, būdama privalomais vartais tarp PĮ kūrėjų ir \emph{iPhone} naudotojų, užtikrina, kad kiekvienas taikomosios PĮ vienetas, kuris bus pateikiamas vartotojui, bus veikiantis, veiks korektiškai ir teisingai, darys tai, ką žada, ir nedarys nieko, ko vartotojas nenorėtų (neįdiegs kenksmingos PĮ, nevogs vartotojo privačių duomenų ir pan.) Kai kurie autoriai \cite{techcrunch_apples_app_store_the_new_walled_garden} šią \emph{Apple} pasirinktą metodiką vadina „sodu, aptvertu aukšta siena“.

Kita vertus, PĮ telefonui \emph{iPhone} kuriantieji kritikuoja \emph{Apple} pasirinktą PĮ modelį, nes šis apriboja jų laisvę. „Paprasti“ (ne \emph{Apple} bendrovės darbuotojai) negali panaudoti kai kurių telefono suteikiamų funkcijų – savo kuriamose programose leidžiama išnaudoti tik tas galimybes, prie kurių priėjimą yra suteikusi \emph{Apple}. Taip pat kūrėjas, sukūręs programą, privalo nusiųsti ją \emph{Apple} bendrovei tam, kad ši programą išbandytų ir patvirtintų; toks patvirtinimas paprastai užtrunka nuo trijų dienų iki mėnesio. Ilgas priverstinis laiko tarpas, vertimas laukti tarp programos sukūrimo fakto ir jos publikavimo datos neleidžia kūrėjams greitai reaguoti į pasikeitimus, taisyti iškilusias klaidas, bandyti įvairesnes idėjas.

Papildomai kiekvienas, norintis publikuoti savo kuriamą PĮ \emph{App Store}, privalo mokėti metinį 99 JAV dolerių (apie 283 Lt) mokestį – už \emph{Apple} teikiamas programų testavimo, publikavimo ir reklamavimo paslaugas. Taip pat \emph{Apple} bendrovė pasiima 30 proc. kainos nuo kiekvienos vartotojui parduotos kūrėjo programos.

Tai žiniasklaidai sukelia nepatogumų, kuriuos kai kurie kritikai nepabijo pavadinti net cenzūra ir grėsme žodžio laisvei \cite{arstechnica_apple_and_app_store_censorship_where_to_draw_the_line}. Vis dėlto, \emph{App Store} šiuo metu yra didžiausia programinės įrangos mobiliesiems telefonams parduotuvė pasaulyje, publikuojanti daugiau nei 65 tūkst. įvairiausių rūšių PĮ vienetų \cite{apple_app_store_downloads_top_1_point_5_billion_in_first_year}, tarp jų – daug programų, sukurtų vienos arba kelių žiniasklaidos bendrovių pateikiamam turiniu vartoti.


\subsection{Žiniasklaidos „mobilumo“ lygiai}

Atskirus žiniasklaidos vienetus (naujienų portalus, laikraščius ar kt.) galėtume išskirstyti pagal tai, kokiais būdais jos suteikia galimybę mobiliojo telefono naudotojui pasiekti savo turinį. Pagal tai žiniasklaidą galėtume skirstyti į šiuos lygius:

\begin{enumerate}
	\item \textbf{Nulinis lygis:} žiniasklaida būdo turiniui pasiekti mobiliuoju įrenginiu specialiai nekuria.
	\item \textbf{Pirmasis A lygis:} žiniasklaida publikuoja mobiliajam įrenginiui pritaikytą tinklalapį savo kuriamam turiniui pasiekti.
	\item \textbf{Pirmasis B lygis:} žiniasklaida yra sukūrusi ir išnaudoja specialią, į mobilųjį įrenginį įdiegiamą programinę įrangą savo kuriamam turiniui pasiekti.
	\item \textbf{Antrasis lygis:} žiniasklaida kartu publikuoja ir tinklalapį, pritaikytą mobiliajam įrenginiui, ir išnaudoja specialią į mobilųjį įrenginį įdiegiamą programinę įrangą savo kuriamam turiniui pasiekti.
\end{enumerate}


\subsubsection{Nulinis lygis}

Šiame lygyje esančios žiniasklaidos bendrovės specialiai nekuria jokių papildomų priemonių tam, kad vienaip ar kitaip įgalintų savo vartotojus (skaitytojus, klausytojus ar žiūrovus) pasiekti savo kuriamą turinį mobiliojo įrenginio pagalba.

Šiam lygiui priklauso daugiausia tos žiniasklaidos bendrovės, kurios dėl kurių nors priežasčių (pavyzdžiui, finansinių) neturintys žiniatinklio tinklalapio internete. Pastebima tendencija, kad į mobiliosios žiniasklaidos erdvę patys pirmieji skverbiasi būtent didesniosios žiniasklaidos bendrovės.

\paragraph{Nulinis lygis siejasi su pirmuoju A lygiu}

Pastebėtina, kad dalis naujesnių mobiliųjų įrenginių, pavyzdžiui, išmanieji telefonai (angl. \emph{smartphone}), geba atidaryti ir panaudoti „įprastą“, mobiliajam telefonui specialiai nepritaikytą tinklalapį internete.

Speciali programinė įranga, sukurta ir įdiegta pačio mobiliojo įrenginio gamintojo (pvz. \emph{Nokia} telefonų integruotosios žiniatinklio naršyklės) arba vartotojo, pasinaudojusio trečiosios šalies – bendrovės, gaminančios specialias žiniatinklio naršykles mobiliesiems telefonams – produktu (pavyzdžiui, \emph{Opera Mini} \cite{opera_mini_and_mobile}, kuriamu Norvegijos kompanijos \emph{Opera Software ASA}), leidžia atverti didesnei ekrano raiškai, pilnai, o ne vien skaičių klaviatūrai, ir kitokiai naršymo aplinkai pritaikytą tinklalapį internete. Šio tinklalapio puslapiai (tekstai, grafiniai elementai) yra sumažinami, viena ar kita metodika „įspraudžiami“ į nedidelį mobiliojo įrenginio ekraną, sukuriama aplinka paprastesnei navigacijai, kurioje neegzistuoja personaliniams kompiuteriams įprastas pelės įrenginys.

Taigi, žiniasklaidos bendrovės, kurios specialiai neruošia savo internetiniu tinklalapių „mobiliajam vartotojui“, paprastai vis tiek susilaukia šiokios tokios dalies „mobiliųjų naršytojų“, kurie žiniasklaidos internetinį resursą vartoja (arba tik bando vartoti) naudodamiesi programine įranga, specialiai sukurta tam, kad mažintų prarają tarp ribotų mobiliųjų įrenginių ir „pilnaverčių“ tinklalapių internete.


\subsubsection{Pirmasis A lygis}

Pirmajame A lygyje yra daugiausia žiniasklaidos bendrovių, pasirinkusių išbandyti mobiliąją erdvę. Šios bendrovės savo skaitytojams (klausytojams, žiūrovams) sukuria ir publikuoja specialų žiniatinklio tinklalapį, pritaikytą peržiūrai mobiliuoju įrenginiu.


\subsubsection{Pirmasis B lygis}

Šis žiniasklaidos turinio „mobilumo“ lygis pasižymi tuo, kad šiuo atveju žiniasklaidos bendrovė yra dėl vienokių ar kitokių priežasčių „apėjusi“ mobiliojo tinklalapio kūrimo galimybę, užtad sukūrusi atskirą įdiegiamą programą pasirinktam mobiliajam įrenginiui, kurią vartotojai gali parsisiųsti, įsidiegti ir vėliau jos pagalba vartoti žiniasklaidos kuriamą turinį.

Šiuo keliu nėra nuėjusios daug žiniasklaidos bendrovių – tendencija yra tokia, kad pirmiausia žiniasklaida (ar bet kurios kitos bendrovės, kurių pagrindinis arba šalutinis pajamų šaltinis yra interneto vartotojai) išbando „pirmąjį A lygį“, t.y. susikuria interneto tinklalapį, pritaikytą mobiliajam telefonui.


\subsubsection{Antrasis lygis}

Šiame lygyje esančios žiniasklaidos bendrovės naudoja abu būdus – specialų tinklalapį ir programą mobiliesiems telefonams – savo auditorijai pasiekti.


\subsection{Lietuvos žiniasklaidos „mobilumas“}

Lietuvos žiniasklaida taip pat išnaudoja mobiliojo interneto galimybes ir teikia prieigą prie savo turinio mobiliaisiais įrenginiais.

Nedideliam tyrimui pasirinkti 10 populiariausių (pagal 2010 m. balandžio „Gemius Audience“ duomenis \cite{gemiusaudience_2010_april_by_reach}) Lietuvos naujienų portalų (žr. \ref{fig:populiariausi_portalai_ir_mobilumo_lygiai} lentelę).

\begin{figure}[h!]
	\caption{Populiariausių Lietuvos internetinių žiniasklaidos bendrovių naudojami mobilieji sprendimai ir šių bendrovių „mobilumo lygiai“.}
	\label{fig:populiariausi_portalai_ir_mobilumo_lygiai}
	\centering
		\begin{tabular}{llllll}
			
			% Antraštė
			Naujienų portalas					&
			XHTML mobilusis tinklalapis			&
			WML mobilusis tinklalapis			&
			\emph{Apple iPhone} programa		&
			\emph{Nokia Web Runtime} programa	&
			„Mobilumo lygis“					\\
			
			% 1. delfi.lt
			1. delfi.lt		&
			Yra				&
			Yra				&
			Nėra			&
			Nėra			&
			1A				\\
			
			% 2. lrytas.lt
			2. lrytas.lt	&
			Yra				&
			Yra				&
			Yra				&
			Yra				&
			2				\\
			
			% 3. balsas.lt
			3. balsas.lt	&
			Nėra			&
			Yra				&
			Yra				&
			Nėra			&
			2				\\
			
			% 4. 15min.lt
			4. 15min.lt		&
			Yra				&
			Nėra			&
			Yra				&
			Nėra			&
			2				\\
			
			% 5. alfa.lt
			5. alfa.lt		&
			Yra				&
			Yra				&
			Nėra			&
			Nėra			&
			1A				\\
			
			% 6. zebra.lt
			6. zebra.lt		&
			Nėra			&
			Yra				&
			Nėra			&
			Nėra			&
			1A				\\
			
			% 7. diena.lt
			7. diena.lt		&
			Nėra			&
			Nėra			&
			Nėra			&
			Nėra			&
			0				\\
			
			% 8. ve.lt
			8. ve.lt		&
			Nėra			&
			Nėra			&
			Nėra			&
			Nėra			&
			0				\\
			
			% 9. panele.lt
			9. panele.lt	&
			Nėra			&
			Yra				&
			Nėra			&
			Nėra			&
			1A				\\
			
			% 10. vz.lt
			10. vz.lt		&
			Yra				&
			Nėra			&
			Nėra			&
			Yra				&
			2				\\
			
		\end{tabular}
\end{figure}


\subsubsection{delfi.lt}

Didžiausias (pagal lankomumą) naujienų portalas Lietuvoje priklauso „pirmajam A lygiui“. „DELFI“ publikuoja žiniatinklio tinklalapį adresu <\url{http://m.delfi.lt}>, taip pat teikia gana ribotą priėjimą prie dalies savo turinio (kai kurių straipsnių) adresu <\url{http://wap.sp.lt}>. Specialios programinės įrangos, sukurtos mobiliajam telefonui ir skirtos savo turiniui vartoti, naujienų portalas nėra publikavęs.

\subsubsection{lrytas.lt}

Antras pagal lankomumą Lietuvos naujienų portalas priklauso „antrajam lygiui“. Adresu \url{http://m.lrytas.lt} yra publikuojamas XHTML technologija sukurtas mobiliojo žiniatinklio tinklalapis, o adresu \url{http://wap.lrytas.lt} – WML technologija sukurtas tinklalapis, skirtas senesniems telefonų modeliams. Taip pat „lrytas.lt“ yra sukūręs specialią programinę įrangą telefonui \emph{iPhone} (\url{http://itunes.apple.com/lt/app/lrytas-lt/id351651546?mt=8}), taip pat – programą telefonams, palaikantiems \emph{Nokia Web Runtime} technologiją (\url{http://store.ovi.com/content/16687}).

\subsubsection{balsas.lt}

„Balsas.lt“ priklauso „antrajam lygiui“. Naujienų portalas pirmasis Lietuvoje išleido programą telefonui iPhone, skirtą portalui naršyti (\url{http://itunes.apple.com/lt/app/balsas-lt/id295219692?mt=8}). Taip pat „balsas.lt“ publikuoja mobilųjį tinklalapį adresu \url{http://wap.balsas.lt}.

\subsubsection{15min.lt}

Naujienų portalas, „išaugęs“ iš laikraščio „15 min“, priklauso „antrajam lygiui“. „15min.lt“ kuria ir palaiko programą, skirtą telefonui iPhone (\url{http://itunes.apple.com/lt/app/15min/id322573191?mt=8}), taip pat palaiko tinklalapį, skirtą mobiliajam telefonui, adresu \url{http://m.15min.lt}.

\subsubsection{alfa.lt}

„Alfa.lt“ priklauso „pirmajam A lygiui“. Naujienų portalas nėra publikavęs jokios programinės įrangos, skirtos tinklalapio turiniui pasiekti, užtad palaiko mobiliuosius tinklalapius adresais \url{http://wap.alfa.lt} (skirtas senesniems, vien WAP technologiją palaikantiems telefonams) ir \url{http://m.alfa.lt} (skirtas naujesniems, XHTML technologiją palaikantiems telefonams).

\subsubsection{zebra.lt}

„Zebra.lt“ taip pat priklauso „pirmajam A lygiui“. Nors programinės įrangos naujienų portalas neteikia, jo turinį iš mobiliojo įrenginio galima pasiekti adresu \url{http://wap.zebra.lt}.

\subsubsection{diena.lt}

Visi „Diena Media Group“ grupei priklausantys naujienų portalai „Vilniaus diena“ (\url{http://vilniaus.diena.lt}), „Kauno diena“ (\url{http://kauno.diena.lt}) ir „Klaipėda“ (\url{http://klaipeda.diena.lt}) priklauso „nulinei grupei“ – nei specialios programinės įrangos, nei naujienų tinklalapio mobiliesiems telefonams nekuria ir nepalaiko.

\subsubsection{ve.lt}

Klaipėdos ir apskritai Vakarų Lietuvos dienraštis „Vakarų ekspresas“ taip pat priklauso „nuliniam lygmeniui“ – nei programos, nei tinklalapio mobiliesiems įrenginiams nepublikuoja.

\subsubsection{panele.lt}

Pramogų portalas „Panele.lt“, papildantis savo pirmtaką „Panelė“, priklauso „pirmajam A“ lygiui – publikuoja vien tinklalapį mobiliesiems telefonams adresu \url{http://wap.panele.lt}. Vis dėlto, tinklalapio bandymo metu \footnote{Mobilusis tinklalapis bandytas 2010 m. birželio 11 d.} jis neveikė.

\subsubsection{vz.lt}

Naujienų portalas „Verslo žinios“ priklauso „antrajam lygiui“ – kartu kuria naujienų portalą mobiliesiems telefonams (\url{http://m.vz.lt}) bei \emph{Nokia Web Runtime} technologiją naudojantį įdiegiamą valdiklį – specialią programą (\url{http://store.ovi.com/content/25180}).


\subsection{Mobiliųjų žiniasklaidos sklaidos kanalų apmokestinimas ir reklama juose}

Vienas svarbiausių klausimų, į kuriuos nori sau atsakyti žiniasklaidos bendrovė, svarstydama ar ir kaip pateikti savo kuriamą turinį kuriuo nors iš mobiliųjų kanalų, yra apmokestinimo ir reklamos klausimas.

Nors į papildomą turinio sklaidą mobiliesiems įrenginiams gali būti žiūrima vien per prekės ženklo plėtros ir žinomumo prizmę, t.y. nesitikima iš sukurtų mobiliųjų sprendimų uždirbti tiesiogiai, yra svarstytinų būdų ir net sėkmės istorijų, kaip žiniasklaidos bendrovės gali iš mobilaus kanalo gauti papildomų pajamų.


\subsubsection{Reklama}

Vienas iš pagrindinių šiandieninės žiniasklaidos pajamų šaltinių yra reklama, įterpiama šalia žiniasklaidos kuriamo turinio. Elektroninėje (internetinėje) žiniasklaidoje reklama paprastai viešinama reklaminių skydelių (angl. \emph{banner}) pavidalu.

\paragraph{Reklamos įdiegimo specifika}

\subparagraph{Mobiliuosiuose tinklalapiuose}

Žiniasklaidai kuriant tinklalapį mobiliesiems įrenginiams, klasikinės elektroninės reklamos – reklaminių skydelių – panaudojimas yra paprasčiausias metodas, kurį gali pasitelkti žiniasklaidos bendrovė.

Paprastumas išplaukia iš tinklalapio kūrimo bei veikimo specifikos – mobilusis tinklalapis naudoja tas pačias (arba bent jau labai panašias) technologijas (HTML, XHTML, CSS, JavaScript) į tas, kurios naudojamos personaliniams kompiuteriams skirtų tinklalapių turiniui ir reklamai parodyti. Mobiliojo tinklalapio arba reklamos sistemos kūrėjams paprastai tenka tik pritaikyti reklaminio skydelio dimensijas (aukštį ir plotį) bei dydį (perduodamų duomenų kiekį), o pats skydelio veikimo principas lieka toks pats.

\subparagraph{Įdiegiamose programose}

Sudėtingiau tenka elgtis, jei norima reklaminį skydelį įterpti į kuriamą specialią įdiegiamą programą, skirtą žiniasklaidos bendrovės kuriamam turiniui vartoti. Reklamos įterpimas į programą dažnai (bet ne visada) reikalauja papildomo darbo realizuojant skydelio parsiuntimo, pozicionavimo, parodymo ir atliekamos funkcijos (reklamuojamojo tinklalapio atidarymo, taško žemėlapyje parodymo ar kt.) vykdymą. Šiam darbui atlikti reklamos sistemos kūrėjui neužtenka vien tinklalapio kūrimui naudojamų aukščiau išvardintų technologijų išmanymo – šis privalo susipažinti ir su mobiliojo telefono programinės įrangos kūrimo priemonėmis (reikalaujama programavimo kalba ir kt.) Tai reikalauja papildomo laiko ir investicijų, o šie resursai gali būti nesuteikti spėjant, kad investicijos grąža (angl. \emph{return of investment}) bus per maža arba net neigiama.

\paragraph{Lokalių reklamdavių trūkumas}

Dar viena priežastis, dėl kurios žiniasklaida kol kas tik bandosi mobiliąją erdvę, yra ta, kad, pavyzdžiui, Lietuvoje, lokalių (vietinių – regiono arba šalies prasme) reklamos sistemų, kurios palaikytų įvairiopus reklamos mobiliuosiuose įrenginiuose pateikimo būdus (būtų pasiruošusios reklamuotis mobiliojoje edvėje), yra labai nedaug, ir šios sunkiai negali iš žiniasklaidos bendrovės „nupirkti“ tiek daug reklamos ploto, kad pastaroji „pajustų“ pajamas, atnešamas savo sukurto mobiliojo sprendimo.

Reklamos sistemų, pasiruošusių mobiliesiems įrenginiams, užtektinai yra globalioje rinkoje, bet jos neturi pakankamo kiekio „vietinės“ reklamos, įdomios vartotojui iš kurio nors konkretaus regiono. Taip pat žiniasklaidai ir reklamdaviams bendradarbiaujant globalesniu mastu, gali iškilti kitų – apmokėjimo, komunikacijos – problemų.

\paragraph{„Adobe Flash“ skydelių dilema}

Nors bene visi (išskyrus itin senus modelius) šiuo metu rinkoje esančių mobiliųjų įrenginių geba atvaizduoti ir panaudoti paprastą statinę reklaminio skydelio iliustraciją (pavyzdžiui, \emph{.JPEG} arba \emph{.PNG} duomenų formato) arba nesudėtingą, techniškai ribotą animaciją (pavyzdžiui, \emph{.GIF} duomenų formato), reklamos rinkoje vienas populiariausių formatų, kuriais pateikiami reklaminiai skydeliai, yra \emph{Adobe Flash} formatas. Šis tinklalapio turinio duomenų formatas leidžia tinklalapyje naudoti įvairesnę, greitesnę ir didesnės raiškos animaciją, realizuoti interaktyvius elementus, rodyti vaizdo medžiagą ir kt. \cite{adobe_what_is_flash_professional}

Reklamdavių, o dėl to – ir žiniasklaidos bendrovių, norą įžengti į mobiliojo interneto rinką gali sustabdyti tai, kad ne visi mobilieji įrenginiai palaiko minėtąjį \emph{Adobe Flash} formatą, o ir tuose įrenginiuose, kuriuose formato palaikymas yra realizuotas, jo veikimas nėra efektyvus (per lėtas, per mažai pritaikytas mažesniam mobiliojo įrenginio ekranui). Taip reklamdaviai lyg ir yra priversti mobiliuosiuose įrenginiuose likti prie „senesnio modelio“ – nuobodesnės, bet labiau suderinamos ir stabiliau veikiančios reklamos formos (statinių reklaminių skydelių).


\subsubsection{Turinio apmokestinimas}

Šis metodas yra sunkiau ir sudėtingiau realizuojamas turint vien tik mobiliesiems įrenginiams pritaikytą specialų tinklalapį. To priežastis yra ta, kad iki šiol nėra sukurtas, išplėtotas ir išpopuliarintas metodas, kaip vartotojai galėtų susimokėti už suvartojamą turinį, pateikiamą interneto tinklalapyje, todėl tokia praktika beveik ne(be)egzistuoja.

Kiek kitokia situacija yra specialių programų mobiliesiems telefonams, skirtų žiniasklaidos turiniui vartoti, rinkoje. Geru pavyzdžiu galėtume pasirinkti jau minėtąjį \emph{App Store} – mobiliesiems įrenginiams \emph{Apple iPhone} ir \emph{Apple iPad} skirtos PĮ parduotuvę internete. Nors parduotuvėje yra nemokamai publikuojama nemažai garsių ir visuotinai pripažintų leidinių sukurtų programų, kai kurioms žiniasklaidos bendrovėms pasiseka analogišką programą parduoti už pinigus, o ne dalinti savo skaitytojams (klausytojams, žiūrovams) ją nemokamai.

\paragraph{„The Guardian“ programos \emph{Apple iPhone} įrenginiui pavyzdys}

Viena iš „sėkmės istorijų“ galėtų būti Jungtinės Karalystės laikraščio ir naujienų portalo „The Guardian“ išleista įdiegiama programa, skirta telefonui \emph{Apple iPhone}. Mokama programa (kainuojanti 2.39 svarus sterlingus, arba apie 10 litų) per du mėnesius buvo nupirkta per 100 tūkst. kartų. \cite{guardian_more_than_100k_downloads_of_guardian_iphone_app} Atsižvelgiant į tai, kad už bet kokios programos publikavimą \emph{Apple} bendrovė sau pasilieka 30 proc. kainos nuo pardavimų, galėtume sakyti, kad „The Guardian“ už programą gavo 700 tūkst. litų papildomų pajamų.



\section{Vartotojas mobiliaisiais kanalais teikia turinį žiniasklaidai}

Žiniasklaidos bendrovei turinį pasufleruoti, o neretai – ir kurti, gali ir pavieniai jos vartotojai, pasinaudodami mobiliaisiais kanalais.

Vartotojo norą pačiam parašyti, nufotografuoti, nufilmuoti ir savo sukurtą produkciją atiduoti žiniasklaidos bendrovei lemia viešumo faktorius: pilietiškų arba asmeninių paskatų vedamas, žiniasklaidos turinio vartotojas gali siekti, kad jo iškelta problema būtų paviešinta žiniasklaidoje, kuria jis pasitiki ir kurią mėgsta \cite{ojr_what_is_participatory_journalism}.


\subsection{Žiniasklaidos kvietimai vartotojams kurti ir siųsti turinį}

Nors, atrodytų, elektroninio pašto laišką pasirinktai redakcijai galėtų nusiųsti kiekvienas ir bet kada, papildomas skaitytojų (klausytojų, žiūrovų) skatinimas iš žiniasklaidos bendrovės pusės tai daryti – pavyzdžiui, kelti sau arba savo šeimai, namui, gyvenvietei, regionui opias problemas, fiksuoti ne vietoje pastatytus automobilius, netvarką miestuose, valstybės pareigūnų netinkamai vykdomas pareigas ir kt. – žymiai padidina tokių pranešimų redakcijai skaičių.

Dalis didžiųjų Lietuvos žiniasklaidos bendrovių bando naudoti „pilietinę žurnalistiką“ ir savo leidiniuose, naujienų portaluose ar kitur papildomai kviečia skaitytojus (klausytojus, žiūrovus) kreiptis į redakciją bei siūlyti temas. Tarp pavyzdžių galėtume paminėti naujienų portalo „DELFI“ puslapį \emph{„Pranešk naujieną“} (\url{http://www.delfi.lt/help/hint.php}), naujienų portalo „15min.lt“ sub–projektą \emph{„Įkrauk“} (\url{http://ikrauk.15min.lt}) arba dar visai „jauno“ Šiaulių naujienų portalo „etaplius.lt“ puslapį \emph{„Atsiųsk naujieną“} (\url{http://www.etaplius.lt/newsadd/add}).

Mobiliosios technologijos yra itin pravarčios žiniasklaidos bendrovėms, siekiančioms gauti pranešimų apie naujienas iš savo vartotojų. Pirmiausia, mobilusis įrenginys labai dažnai arba visada būna šalia vartotojo, todėl vartotojas, pastebėjęs, jo nuomone, žiniasklaidos dėmesio vertą įvykį, gali tą pačią minutę apie jį pranešti – jam nereikia sau susiplanuoti vėlesnio pranešinėjimo žiniasklaidai ir šitaip skirti papildomų pastangų, taip neretai sumažinant savo motyvaciją tai daryti apskritai.

Antra, šiuolaikiniame mobiliajame įrenginyje (tarkime, mobiliajame telefone) dažnai būna beveik visa reikalinga techninė įranga, panaudotina neprofesionaliam, mėgėjiškam įvykio nušvietimui – mobilusis telefonas moka fotografuoti, filmuoti, įrašyti garsą, juo galima kurti neilgus tekstus; svarbiausia, kad telefonas yra prijungtas prie interneto, taigi iškart, iš pastebėtojo įvykio vietos pranešimą galima iškart ir išsiųsti.

Šiuo metu Lietuvos žiniasklaidos erdvėje galimybę pranešti redakcijai apie pastebėtą įvykį mobiliuoju telefonu naudoja bei skatina tik naujienų portalas „15min.lt“, savo programoje \emph{15min}, skirtoje \emph{Apple iPhone} mobiliajam įrenginiui, teikdamas funkciją pavadinimu „Įkrauk“. Pasirinkęs minėtąją funkciją, vartotojas gali aprašyti pastebėtą įvykį, įkelti savo padarytas įvykio nuotraukas ir visa tai nusiųsti „15min.lt“ redakcijai. Papildomai vartotojo prašoma nurodyti savo vardą bei telefono numerį, jeigu redakcijai reikėtų paskambinti pranešusiajam ir patikslinti atsiųstą informaciją.

Nors kitų portalų sukurtos priemonės (pavyzdžiui, DELFI „Pranešk naujieną“, etaplius.lt „Atsiųsk naujieną“) taip pat gali būti atvertos bei panaudotos mobiliuoju telefonu, jos tam nebuvo specialiai kurtos ir nėra specialiai pritaikytos naudojimui mobiliajame įrenginyje.


\subsection{\emph{USTREAM} ir pedofilijos skandalo atvejis}

Įdomus pilietinės žurnalistikos, kurioje naudojamasi mobiliųjų įrenginių ir mobiliojo interneto teikiamomis galimybėmis, pavyzdys buvo mobiliuoju telefonu filmuojama ir tiesiogiai internetu transliuojama Nijolės Venckienės, vienos iš vadinamosios „pedofilijos skandalo“ dalyvės, namo Garliavoje apsuptis.

Vadinamasis pedofilijos skandalas prasidėjo nuo to, kad kaunietis Drąsius Kedys viešai išplatino įtarimus apie savo nepilnametės dukters prievartavimą su jos motinos, Laimos Stankūnaitės, žinia. Vėliau Kaune buvo nužudytas vienas iš kelių įtariamųjų prievartautojų; žudynių įtariamuoju tapo D. Kedys. Vėliau, radus negyvo D. Kedžio kūną, Kėdainių rajono apylinkės teismas nusprendė, kad D. Kedžio dukra, laikotarpiu tarp žudynių ir jos tėvo kūno radimo apgyvendinta pas D. Kedžio seserį Nijolę Venckienę, privalo būti grąžinta savo motinai L. Stankūnaitei \cite{delfi_teismas_kedzio_dukrele_grazino_motinai}.

Šis teismo sprendimas susilaukė daug priešiškų ir emocionalių vertinimų, turėjo didelį atgarsį Lietuvos visuomenėje. Tą pačią dieną, kai buvo priimtas sprendimas, prie N. Venckienės namų Garliavoje (Kauno rajone) ėmė rinktis su sprendimu nesutinkantys žmonės, siekę budėti prie namo ir neleisti išgabenti D. Kedžio dukros iš N. Venckienės namų pas L. Stankūnaitę. Šie namo apsuptį vykdė apie savaitę \cite{delfi_teismas_sustabde_sprendima_perduoti_kedzio_dukra_motinai_stankunaitei}.

Apsuptį mobiliuoju telefonu filmavo ir „gyvai“ internete transliavo asmuo, transliacijų tinklalapyje \emph{USTREAM} (\url{http://www.ustream.tv}) pasivadinęs slapyvardžiu \emph{negyvas} (slapyvardis greičiausiai pasirinktas kaip aliuzija į tuo metu nesenai rastą D. Kedžio kūną). \emph{negyvas} tinklalapyje sukūrė kanalą pavadinimu \emph{Kaunas-Vencku-Namai} (\emph{sic}), adresu \url{http://www.ustream.tv/channel/kaunas-vencku-namai}; tuomet į šį kanalą, naudodamasis viena iš specialių \emph{USTREAM} sukurtų programų mobiliesiems įrenginiams, filmavo ir tiesiogiai transliavo įvykius, vykusius prie N. Venckienės namų.

\emph{USTREAM} yra nemokamas vaizdo medžiagos platinimo tinklalapis internete, nuo kitų (pavyzdžiui, \emph{YouTube}) besiskiriantis tuo, kad šio vartotojai yra skatinami ne įkelti nufilmuotus ir paruoštus video įrašus, o filmuoti „gyvai“ ir transliuoti šį vaizdą internetu. \emph{USTREAM} tam yra sukūrusi keletą programų įvairiems populiariems mobiliųjų telefonų modeliams. Tinklalapio lankytojas, norėdamas pradėti transliaciją, turi nemokamai užsiregistruoti tinklalapyje, tuomet į savo mobilųjį telefoną įdiegti tinkančią \emph{USTREAM} sukurtą filmavimo ir transliavimo programą, ir iškart gali pradėti transliuoti.

\emph{negyvas} pats vienas sugebėjo sutraukti nemažą auditoriją. \emph{USTREAM} tinklalapis skelbia, kad iš viso gyva transliacija iš Garliavos buvo internautų įsijungta per 318 tūkst. kartų. Unikalių peržiūrų (unikaliu laikomas vienas iki tol tinklalapio kanale neapsilankęs žmogus, peržiūrėjęs nors vieną video įrašą arba transliaciją 24 val. laikotarpiu) buvo beveik 58 tūkst. Iš viso visi lankytojai kartu sudėjus tiesioginę \emph{negyvas} vartotojo kuriamą transliaciją žiūrėjo 4157 dienas, arba 99768 valandas. Prie \emph{negyvas} populiarumo prisidėjo ir kiti internautai, aktyviai dalinęsi nuoroda į Garliavos įvykius tiesiogiai transliuojantį \emph{USTREAM} kanalą – paieškos sistema „Google“, ieškojus 2010 m. birželio 12 d., internete rado apie 2600 nuorodų, vedančių \emph{negyvas} sukurtojo kanalo adresu \cite{google_search_negyvas_url}.



%
% Išvados
%

\vukfConclusion

Žiniasklaidoje pastebimos dvi populiariausios praktikos savo turiniui mobiliaisiais kanalais pateikti. Pirmoji yra specialaus mobiliojo tinklapio sukūrimas. Šioji labiausiai paplitusi, nors ir turi tam tikrų techninių apribojimų, be to, ją dėl aparatūrinės technikos skirtumų kartais gali būti sunku realizuoti. Antroji paplitusi praktika yra įdiegiamos programinės įrangos, skirtos žiniasklaidos kuriamam turiniui vartoti, sukūrimas ir publikavimas. Tokia PĮ yra mažiau ribota galimybėmis, bet kenčia nuo to, kad šiai kurti reikia įgauti arba turėti daugiau specifinių žinių, taip pat nuo to, kad tarp PĮ kūrėjo ir vartotojo dažnai „stovi“ papildomas trečiasis asmuo (mobiliojo įrenginio gamintojas), kuris riboja pateikiamo turinio įvairovę.

Lietuvos žiniasklaida taip pat yra žengusi ne tik pirmuosius, bet jau ir antruosius žingsnius į mobiliąją rinką. Vis dėlto, ne visi leidiniai dar spėjo pasivyti nūdienos tendencijas, o ir egzistuojantiems žiniasklaidos sprendimams dar yra ką įvertinti kitaip ir patobulėti.

Patys žiniasklaidos vartotojai dažnai nori ir net siekia teikti savo kuriamą turinį žiniasklaidai – būti pastebėti, publikuoti, pademonstruoti rūpimas bėdas ir kt. „Didžioji“ žiniasklaida Lietuvoje kol kas nedaug išnaudoja šį atgalinio ryšio potencialą. Jai derėtų iš naujo įvertinti vartotojų sukuriamo turinio teikiamą naudą, o taip pat apsvarstyti galimybę pačiai profesionaliu lygiu naudoti vartotojų įsisavintas turinio kūrimo metodikas (pavyzdžiui, USTREAM tiesiogines transliacijas).

Šiuo darbu tiriamasis objektas (mobilioji žiniasklaida) buvo tik „apžiūrėtas iš išorės“. Tolimesniai tyrimai šia kryptimi galėtų išsiaiškinti žiniasklaidos bendrovių požiūrį į mobiliuosius kanalus, patikslinti žiniasklaidos skaitytojų / klausytojų / žiūrovų vartojimo įpročius ir motyvus, taip pat panagrinėti, kokiu būdu žiniasklaida pati naudoja mobiliuosius kanalus profesionalios informacijos rinkimui.


%
% Bibliografija
%

\bibliography{kursinis-ziniasklaida-mob-bibliografija}

\end{document}
