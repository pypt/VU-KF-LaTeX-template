\documentclass[kursinis-darbas]{vukf}

\begin{document}
	
\title{Nelegaliai veikiančios radijo stotys: istorija, transliuotojai ir klausytojai}
\vukfTitleEn{Illegally broadcast radio stations: history, broadcasters, and listeners}
\author{Linas Valiukas}
\vukfAuthorFirstName{Linas}
\vukfAuthorLastName{Valiukas}
\vukfAuthorDescription{Žurnalistikos bakalauro studijų programos \\ III kurso II grupės studentas}
\vukfDivision{Žurnalistikos institutas}
\vukfDivisionEn{Institute of journalism}
\vukfKeywords{radijas, piratinis radijas, nelegalios radijo transliacijos, Didžioji Britanija, JAV, Izraelis, mikroradijas, slaptasis radijas, propaganda}
\vukfKeywordsEn{radio, pirate radio, illegal radio broadcast, Great Britain, US, Israel, micro--radio, clandestine radio, propaganda}
\vukfSupervisor{lekt. Giedrė Čiužaitė}

\maketitle
\tableofcontents


\begin{vukfAbstract}
	
% **Referato lape** pateikiamas bibliografinis kursinio darbo aprašas (lygia greta nurodomas jo pavadinimas anglų kalba), pagrindiniai žodžiai (lietuvių, anglų kalba) ir kursinio darbo santrauka (lygiagretus tekstas lietuvių ir anglų kalba).

Šis kursinis darbas analizuoja nelegalių (piratinių) radijo stočių istoriją, dabartį, atskirus jų tipus, motyvus transliuoti šiuo būdu ir nelegalių transliacijų pasekmes.

Aptariamos pirmosios (ankstyvosios) nelegaliai transliuojamos ofšorinės radijo stotys Europoje (Didžiojoje Britanijoje bei Skandinavijoje), jų padaryta įtaka radijo industrijai. Taip pat nagrinėjami dabartinių piratinių radijo stočių veiklos metodai, jų konfliktas su radijo ryšį reguliuojančiomis įstaigomis.

Apžvelgiami kelios skirtingos motyvacijos organizuoti nelegalias radijo transliacijas (komercinis motyvas Didžiojoje Britanijoje, pilietinis bei taikdariškas motyvas Jungtinėse Amerikos Valstijose ir Izraelyje, karinis motyvas įvairių valstybių atžvilgiu). Svarstoma, ar nelegalios radijo transliacijos išnyks išpopuliarėjus internetinei erdvei.

\end{vukfAbstract}

\begin{vukfAbstractEn}

This study analyzes a case of illegal (non--licensed) radio broadcasting (pirate radios). The study reviews the pirate radio history, discusses a contemporary pirate radio, attempts to showcase different types of such radio stations, reviews various actual motives for broadcasting such a type of a radio station, and also pinpoints consequences of broadcasting illegally.

The first illegal offshore radio stations in Europe, mainly in Great Britain and the Scandinavia region, are described, the influence made my them to the whole radio industry is discussed. Also, the inner workings of a contemporary British pirate radio as well as the conflict between the illegal radio station and the radio wave regulator is analyzed.

In the study, several different motivations of running a pirate radio are described, including a commercial motive in Great Britain, „public spirited“ and peace--making motive in US and Israel, and a military motive among various nations. A possible future for illegal radio broadcasting in the advent of the internet is discussed.

\end{vukfAbstractEn}


%
% Glosarijus
%

\newacronym{FCC}{FCC}{\emph{Federal Communications Commision} („Federalinė komunikacijų komisija“, nacionalinis JAV transliacijų reguliuotojas)}

\newacronym{Ofcom}{„Ofcom“}{\emph{Office of Communications} (nacionalinis Didžiosios Britanijos transliacijų ir telekomunikacijų reguliuotojas)}

\newacronym{BBC}{BBC}{\emph{British Broadcasting Corporation} (nacionalinis Didžiosios Britanijos radijo ir televizijos transliuotojas)}


%
% Įvadas
%

\vukfIntroduction

% **Įvade** pristatoma darbo tema, aptariamas ir pagrindžiamas darbo aktualumas, nurodomas kursinio darbo tikslas ir uždaviniai, pristatomi darbe panaudoti tyrimo metodai, apibendrintai aptariama literatūra, atskleidžiama darbo struktūra. Rekomenduojama įvado apimtis – 5–10 proc. kursinio darbo apimties.

Kursinio darbo tema – nelegaliai transliuotos ir transliuojamos radijo stotys, dažnai paprasčiau vadinamos piratinėmis radijo stotimis.

Darbe analizuojamos priežastys transliuoti nelegaliai; aptariami ankstyviausi, žymiausi (Didžiojoje Britanijoje) bei išskirtiniai (Jungtinėse Amerikos Valstijose, Izraelyje) piratinių radijo stočių atvejai. Taip pat nagrinėjami atvejai, kuomet „piratais“ būdavo ne pavieniai privatūs asmenys ar įmonės, bet valstybės („slaptųjų radijų“ atveju). Aptariama ir galima piratinių radijo stočių plėtra (ar persikėlimas) į interneto erdvę.

Nelegalių radijo transliacijų tema aktuali ir tyrėjams, nagrinėjantiems medijų sąsają su žodžio laisve, ir entuziastams, kuriuos masina pasipriešinimo „sistemai“ (angl. \emph{establishment}) tema.\\

\textbf{Kursinio darbo tikslas}: išanalizuoti nelegaliai transliuojamų (piratinių) radijo stočių atsiradimą, raidą, ištirti jų raidos dėsningumus.\\

\textbf{Kursinio darbo uždaviniai}:

\begin{itemize}
	\item išanalizuoti nelegalių radijo transliacijų raidą, rasti sąsajas tarp skirtingų laikotarpių, dėsningumą ir tęstinumą;
	\item kategorizuoti nelegalias radijo transliacijas į atskirus tipus pagal transliuotojus ir jų motyvus;
	\item nustatyti motyvus vykdyti nelegalias radijo transliacijas įvairiomis sąlygomis;
	\item įvardinti žalą, kurią sukelia nelegalios radijo transliacijos.
\end{itemize}

Piratinės radijo stotys daugiausia analizuotos tų šalių, kuriuose jos labiausiai buvo arba yra paplitusios, mokslininkų – Didžiosios Britanijos, JAV, Izraelio. Vis dėlto, įdomią ir intriguojančią temą plačiai aprašė (parodė) žiniasklaida bei nepriklausomi tyrėjai – referuojami ir tokio tipo šaltiniai.

Šis kursinis darbas yra istoriografinio pobūdžio. Jame daugiausia remiamasi Robert Chapman knyga „Selling the Sixties: The Pirates and Pop Music Radio“, Sean Street darbu „Crossing the Ether: British Public Service Radio and Commercial Competition 1922-1945“, Christopher H. Sterling redaguojama „Encyclopedia of Radio“, Carole Fleming „The Radio Handbook“, straipsniais žurnaluose, radijo ir televizijos reportažais. Autoriaus žiniomis, piratinių radijų tema Lietuvoje iki šiol netyrinėta ir neaprašyta; tikimasi, kad šis darbas paskatins tolesnes nelegalių transliacijų studijas.


\section{Apibrėžimas ir termino kilmė}

Piratinio radijo apibrėžimas tirtoje literatūroje dažnai nėra konkretus, tikslus ir vienodas.

Pavyzdžiui, Hugh Chignell piratinį radiją apibrėžia kaip \emph{„nelicencijuotą radijo stotį, kurios sąvoka daugiausia yra istoriškai susijusi su ofšorinėmis radijo transliacijomis Britanijoje 20 amžiaus 7--ajame dešimtmetyje“} \cite[p.~137]{hc_key_concepts_in_radio_studies}. Kita autorė, Carole Fleming, knygoje „The Radio Handbook“ piratinį radiją taip pat apibūdina tiesiog kaip \emph{„nelicencijuotą, nelegalų transliavimą“} \cite[p.~35]{cf_the_radio_handbook}. Panašus apibrėžimas yra nurodytas ir Christopherio H. Sterlingo redaguojamoje „Radijo enciklopedijoje“ („Encyclopedia of Radio“): \emph{„Piratinis radijas yra terminas, skirtas apibūdinti stotis, kurios transliuoja be vyriausybės sutikimo“} \cite[p.~237]{chs_encyclopedia_of_radio}. „Continuum encyclopedia of popular music of the world“ skyrelyje apie piratinį radiją rašo: \emph{„nelegalių radijo transliacijų sinonimas“} \cite[p.~447]{js_continuum_encyclopedia}.

Išsamiausią ir diskrečiausią aptiktą apibrėžimą įvardino Sean Street, kuris piratinį radiją savo darbe „Crossing the Ether: British Public Service Radio and Commercial Competition“ apibūdino kaip \emph{„garso transliavimo formą, pažeidžiančią (transliacijų) licencijavimo reguliavimą (taisykles) arba valstybėje, iš kurios transliuojama, arba valstybėje, į kurią transliuojama, arba abejose.“} \cite[p.206]{ss_crossing_the_ether_british_public_service_radio_and_commercial_competition}. Šis piratinio radijo apibrėžimas yra diskrečiausias, tiksliausias ir išsamiausias nagrinėtoje literatūroje, taigi kursinio darbo autorius toliau remsis juo.

Patogumo ir trumpumo dėlei, šiame darbe nelegaliai transliuojamos radijo stotys bus vadinamos ne tokiu formaliu terminu – \emph{piratinėmis radijo stotimis}, arba tiesiog \emph{piratiniais radijais}.

Tai, kad neteisėtai transliuojamas radijas vadinamas „piratiniu“, reikėtų sieti su tuo, kad pirmosios neteisėtos transliacijos vyko iš laivų, nuplaukusių už valstybės jūrinės sienos į tarptautinius vandenis \cite[p.~447]{js_continuum_encyclopedia}.

Tai patvirtina ir senas (1965 m. publikuotas) N. March Hunnings mokslinis straipsnis „Pirate Broadcasting in European Waters“ \cite[p.~412]{nmh_pirate_broadcasting_in_european_waters_1965}. Autorius nurodo, kad rašymo metu gali suskaičiuoti mažiausiai 11 transliuojančių piratinių radijo stočių, ir paaiškina galimus valstybių vyriausybių motyvus ko nors dėl jų imtis:

\begin{quotation}
	Sunkumai iškyla, nes šios radijo stotys netransliuoja iš kurios nors konkrečios valstybės, besilaikydamos jos įstatymų, transliuoja jau kam nors kitam priskirtais bangų ruožais, bei transliuoja iš tarptautiniuose vandenyse plaukiančių laivų, kurie yra arba išvis be vėliavos, arba su „patogumo vėliava“\footnote{„Patogumo vėliava“ (angl. \emph{flag of convenience}) – situacija, kuomet laivas yra registruotas kitoje nei savininko valstybėje tam, kad šitaip būtų išvengta mokesčių arba kitų įstatyminių prievolių.}.
\end{quotation}

Hunnings taip pat vardina šiuos valstybių motyvus spręsti nelegalių transliacijų problemą:

\begin{enumerate}
	\item Piratinės radijo stotys veikia tuose bangų ruožuose, kurie jau yra teisiškai priskirti tarpvalstybiniais susitarimais, arba pačių valstybių viduje;
	\item Piratinės radijo stotys sukelia pavojų valstybių „transliacijų monopolijai“ ir komercinių transliacijų draudimui;
	\item Piratinės radijo stotys kartais transliuoja „pop“ muziką nesisimokėdamos autorinių mokesčių;
	\item Piratinių radijo stočių savininkai, veiklą vykdydami iš tarptautinių vandenų (nebūdami jokios valstybės jurisdikcijoje), nemoka pajamų ir kitų mokesčių.
\end{enumerate}

Taigi, transliacijos iš už valstybės jūrinės ribos atgal į tą kurią nors valstybę ilgą laiką buvo legalios (nereguliuojamos).

Nors piratinių radijo stočių epicentras buvo (ir išlieka) Didžioji Britanija, pirmosios tokio tipo stotys pradėjo kurtis kitose Europos valstybėse.

Viena iš pirmųjų (galbūt ir pirmoji) iš jūros transliuojama piratinė radijo stotis Europoje buvo daniška – 1958 m. transliacijas į Danijos teritoriją iš Vokietijoje registruoto laivo \cite[p.~447]{js_continuum_encyclopedia} pradėjo „Radio Mercur“ \cite[p.~237]{chs_encyclopedia_of_radio} \cite{hn_radio_mercur_when_mercury_got_wings} \cite[p.~447]{js_continuum_encyclopedia}. Po dviejų metų, 1960 m., sekė taip pat skandinaviška ofšorinė piratinė radijo stotis „Radio Nord“, transliavusi Švedijai. Tais pačiais metais (1960 m.) transliuoti Nyderlandams pradėjo ir „Radio Veronica“, po metų pradėjusi transliacijas ir anglų kalba ir gyvavusi iki 1974 m.

„Radio Mercur“ Danijos vyriausybės iniciatyva buvo gana agresyviai uždaryta 1962 m. \cite[p.~447]{js_continuum_encyclopedia} „Mercur“ laivą (tą patį, kuris registruotas Vokietijoje) perpirko švedų „Radio Syd“ ir iš jo transliavo Švedijai iki 1966 m.


\section{Piratinis radijas Didžiojoje Britanijoje}

Didžiojoje Britanijoje pirmosios piratinės radijo stotys atsirado praėjusio amžiaus šeštojo dešimtmečio viduryje (gerokai prieš pirmąją legalią komercinę radijo stotį) ir labiausiai „vešėjo“ septintajame dešimtmetyje.

„Radijo enciklopedija“ išskiria keturis svarbiausius piratinių radijo stočių atsiradimo Didžiojoje Britanijoje faktorius \cite[p.~237]{chs_encyclopedia_of_radio}:

\begin{enumerate}
	\item \textbf{Besitęsiantis nacionalinio radijo transliuotojo \gls{BBC} monopolis.} Pirmoji komercinė televizija „ITV“ Britanijoje pradėta rodyti 1955 m. \cite{itv_first_commercial_tv_in_britain} (prieš metus buvo priimtas naujas Televizijos įstatymas, leidęs tai padaryti), bet pirmasis (legalus) komercinis radijas gerokai vėlavo – pirmoji komercinė radijo stotis LBC transliuoti pradėjo tik 1973 m., po aštuoniolikos metų.
	\item \textbf{Palankios teisinės sąlygos transliacijoms iš jūros organizuoti.} Jūrinė Didžiosios Britanijos siena baigėsi vos už 3 jūrinių mylių (apytiksliai 5,5 m) nuo kranto, o dėl beveik neegzistavusių tarptautinių susitarimų dėl jurisdikcijos tarptautiniuose vandenyse Didžiosios Britanijos teisė už šios 3 mylių ribos nebegaliojo, taigi radijo stotys galėjo legaliai transliuoti savo programas atgal į Britaniją.
	\item \textbf{Vidiniai nacionalinio transliuotojo \gls{BBC} nesutarimai}, dėl kurių transliuotojas nesugebėdavo „pataikyti“ į jaunesnės auditorijos skonį. Christopher H. Sterling „Radijo enciklopedija“ dvi nacionalinio Britanijos transliuotojo viduje kovojančias grupes vadina „elitistais“ ir „populistais“ \cite[p.~237]{chs_encyclopedia_of_radio}. Pirmieji pasisakė už tradicinį radiją, visuomenės švietimą, antrieji – už „laisvesnę“, populiaresnę programą.
	\item \textbf{Tuomet vyravusi „niūriai senamadiška visuomenės atmosfera“} \cite[p.~237]{chs_encyclopedia_of_radio}. Vyresnioji britų karta buvo ką tik išgyvenusi du pasaulinius karus ir trečiojo dešimtmečio Didžiąją ekonominę krizę. Jaunesnioji karta, savo ruožtu, stengėsi bėgti nuo slogios socialinės aplinkos daugiausia amerikietiškojo rokenrolo pagalba, bet \gls{BBC} šios muzikos transliavo labai nedaug.
\end{enumerate}

Dauguma į Britanijos auditoriją nukreiptų ofšorinių radijo stočių transliavo iš laivų, plaukiojančių Šiaurės jūroje. Darbo sąlygos jūroje buvo pavojingos dėl šalčio, audringo oro ir stipraus vėjo \cite[p.~238]{chs_encyclopedia_of_radio}. Darbo sąlygos tarp dviejų piratinės radijo stoties pusių – vadovybės, paprastai dirbusios brangiame didmiesčio rajono pastate, ir pačių transliuotojų, darbą atlikdavusių pavojingame, šaltame, drėgname laive – labai skyrėsi \cite[p.~238]{chs_encyclopedia_of_radio}.

Transliuoti piratinę radijo stotį buvo pelningas verslas. Šeštajame dešimtmetyje Didžiojoje Britanijoje buvusi 21 radijo stotis \cite[p.~139]{hc_key_concepts_in_radio_studies} priklausė vos šešiems savininkams \cite[p.~238]{chs_encyclopedia_of_radio}.

Manoma, kad 1965 m. pabaigoje Didžiojoje Britanijoje piratinių radijo stočių klausėsi iki 15 mln. klausytojų \cite[p.~447]{js_continuum_encyclopedia} \cite{bbc_radio_4_do_pirates_rule_the_air_waves}; 55 mln. gyventojų turinčioje valstybėje \cite{uk_population_1965} tai reiškė, kad daugiau negu kas ketvirtas gyventojas klausosi nelegalių radijo transliacijų.


\subsection{„Radio Caroline“ (Didžioji Britanija)}

Viena populiariausių ir reikšmingiausių britų piratinių radijo stočių buvo „Radio Caroline“, pradėta transliuoti 1964 m. ir pavadinta metais anksčiau (1963 m.) nužudyto JAV prezidento John F. Kennedy dukros Caroline Bouvier Kennedy vardu \cite[p.~238]{chs_encyclopedia_of_radio}. „Radio Caroline“ su visomis kitomis nelegaliomis radijo stotimis Didžiosios Britanijos vyriausybė uždarė 1967 m.

S. Street taip apibūdina „Radio Caroline“ atsiradimą \cite[p.~202]{ss_crossing_the_ether_british_public_service_radio_and_commercial_competition}:

\begin{quotation}
	1964 m. kovo 29 d., sekmadienį, „Radio Caroline“, jauno airių verslininko Ronan O'Rahilly minties vaisius, pradėjo pirmąsias savo transliacijas tarptautiniuose vandenyse, penkias mylias \emph{(apie ~9,26 km – L.V. past.)} nutolęs nuo Harvičo \emph{(miesto Didžiosios Britanijos salyno rytuose – L.V. past.)}.
	
	Vėliau sekė tokio pobūdžio radijo stočių „sprogimas“, radikaliai pakeitęs radijo jaunimui kultūrą Didžiojoje Britanijoje ir galiausiai privertęs \gls{BBC} 1967 m. įkurti „Radio One“.
	
	Tai buvo „amerikietiško stiliaus“, didžėjaus reprezentuojamas popmuzikos radijas, kuris grojo iki tol Britanijoje negirdėtą ir tiesiogiai \gls{BBC} politikai priešingą muziką.
\end{quotation}

„Radio Caroline“ transliavo iš dviejų radijo stočių „Caroline North“ (angl. „Šiaurės Caroline“) ir „Caroline South“ (angl. „Pietų Caroline“). Pirmoji („Caroline North“) yra laikoma originalesne savo programavimu; ji propagavo naujus populiarius muzikos stilius, pvz. vadinamąjį \emph{merseybeat} („beat“ muzikos porūšį, ypač išpopuliarintą „The Beatles“) ar afroamerikiečių atliekamą muziką.

„Radio Caroline“ reikšmingai prisidėjo prie to, kad radijo stoties didžėjus būtų laikomas tos radijo stoties reprezentaciniu veidu (ar balsu) \cite[p.~238]{chs_encyclopedia_of_radio} \cite[p.~447]{js_continuum_encyclopedia}. 1966 m. sausio 22 d. laikraščio „The Guardian“ numeris apibūdino piratinių radijo stočių formatą kaip \emph{„nesibaigiančią popmuziką ir nereišmingą plepėjimą“} \cite[p.~447]{js_continuum_encyclopedia} – šitoks formatas itin skyrėsi nuo tuo metu vykusių legalių transliacijų pobūdžio ir buvo tarsi prekursorius šiuolaikinių (21 a.) komercinių radijo stočių veiklai.


\subsection{„Radio London“ (Didžioji Britanija)}

„Radio London“ buvo viena iš svarbiausių Britanijos piratinių radijo stočių, kaip ir „Radio Caroline“ transliacijas pradėjusi 1964 m.

Laikoma, kad ši radijo stotis pasitarnavo kaip modelis nacionaliniam transliuotojui \gls{BBC} kuriant popmuzikos radijo stotį „BBC Radio One“. „Radio One“ buvo tiesioginis vyriausybės atsakas piratinėms radijo stotims, su jomis konkuravo \cite[p.~238]{chs_encyclopedia_of_radio} \cite[p.~448]{js_continuum_encyclopedia}. Įdomu, kad \gls{BBC} į „Radio One“ įdarbino keletą tų pačių piratinių radijo stočių didžėjų, pavyzdžiui:

\begin{itemize}
	\item \textbf{Tony Blackburn}. Gimė 1943 m., dirbo „Radio Caroline“ ir „Radio London“; į \gls{BBC} atėjo 1967 m., pradėjo „Radio One“ transliacijas žodžiais \emph{„...and good morning everyone! Welcome to the exciting new sound of Radio 1“}; \gls{BBC} tebedirba iki šiol \footnote{2011 m. rūgpjūtis}) \cite{guardian_radio_1_launches}.
	\item \textbf{John Peel}. Gim. 1939 m., dirbo „Radio London“, į \gls{BBC} „Radio One“ atėjo dirbti 1967 m. Žymus ir įtakingas radijo stoties didžėjus, labiausiai žinomas dėl eksperimentinės laidos „John Peel Sessions“, kuri padėjo išgarsėti daugeliui muzikos stilių ir grupių \cite{bbc_john_peel_sessions}. J. Peel ignoruodavo piratinių radijo stočių nustatytus kūrinių sąrašus ir vietoje to grodavo savo paties pasirinktą muziką; taip pat didžėjus garsus tuo, kad balsu neįsiterpdavo į muzikos kūrinio pradžią ar pabaigą – esą dėl to, kad klausytojui netrukdytų jo įsirašyti turima įrašymo įranga \cite[p.~139]{hc_key_concepts_in_radio_studies}. Mirė 2004 m.
\end{itemize}


\subsection{Ankstyvųjų radijų „aukso amžiaus“ baigtis}

Geriausi laikai Britanijos piratinėms radijo stotims baigėsi 1966 m. vasaros pabaigoje, kuomet Didžiojoje Britanijoje buvo priimtas Jūrinių nusikaltimų įstatymas (angl. \emph{Marine Offences Act}; kitoje literatūroje taip pat vadinamas \emph{Marine etc., Broadcasting (Offences) Bill} \cite[p.~139]{hc_key_concepts_in_radio_studies}), kuris tiesiogiai neuždraudė piratinių radijo stočių, bet neleido Britanijos reklamdaviams jose reklamuotis, taigi nutraukė šių radijo stočių finansavimą \cite[p.~238]{chs_encyclopedia_of_radio}. Įstatymas įsigaliojo 1967 m.

Ankstyvosios piratinės radijo stotys Britanijoje gyvavo neilgai, bet suspėjo padaryti reikšmingą įtaką radijui:

\begin{itemize}
	\item \gls{BBC} radijo stočių eteryje buvo pradėta transliuoti populiarioji (ne vien tradicinė) muzika, įkurta „\gls{BBC} Radio One“ radijo stotis.
	\item Buvo suprasta, kad radijo stoties eteryje galima kalbėti ne vien „kruopčiai artikuliuota bendrine anglų kalba“ \cite[p.~238]{chs_encyclopedia_of_radio}, bet ir smagiai, gyvai plepėti, improvizuoti.
	\item Pirmosioms Britanijos piratininėms radijo stotims reikia dėkoti ir už popmuzikos išpopuliarinimą Europos radijo stočių eteriuose \cite[p.~447]{js_continuum_encyclopedia}.
\end{itemize}


\subsection{Dabartinių laikų Britanijos piratinės radijo stotys}

Nuo septintojo dešimtmečio vidurio, kuomet komercinės ofšorinės piratinės radijo stotys buvo uždraustos Britanijoje, prasidėjo antroji „piratų“ karta – naujieji piratiniai radijai dažniausiai transliavo jau nebe iš jūros, o iš sausumos. Taip pat transliacijų motyvai buvo kitokie \cite[p.~448]{js_continuum_encyclopedia} – pavyzdžiui, „antrosios kartos“ Didžiosios Britanijos „piratai“ transliavo regį, hiphopą ir kitokią „juodųjų muziką“, kurią ignoravo arba apskritai draudė \gls{BBC} ir komerciniai transliuotojai \cite[p.~447]{js_continuum_encyclopedia}.

Šiuolaikinės (21--ojo amžiaus) piratinės radijo stotys Didžiojoje Britanijoje nebėra tokios didelės sklaidos ar populiarumo, kokios buvo ankstyvosios – jos dažniausiai transliuojamos vien kuriame nors mieste ar miestelyje, taip pat dažnai būna skirtos tik kuriam nors vienam socialiniam sluoksniui (etinei mažumai) – išeiviams iš kurios nors vienos valstybės ar tiems, kurie kalba viena iš užsienio kalbų. Dalis iš jų veikia tik savaitgaliais \cite[p.~238]{chs_encyclopedia_of_radio}.

Daugiausia piratinių radijo stočių Didžiojoje Britanijoje šiuo metu\footnote{2011 m.} yra didžiuosiuose miestuose – Londone, Birmingeme, Lydse ir Mančesteryje \cite{ofcom_illegal_broadcasting_factsheet}. Pasak \gls{Ofcom} statistikos, 2010 m. Didžiojoje Britanijoje buvo apie 150 nelegalių radijo stočių. Taip pat – 16\% Londono gyventojų reguliariai jų klausėsi \cite{bbc_radio_4_do_pirates_rule_the_air_waves}. \gls{Ofcom} mano, kad šiandien \footnote{Radijo laida transliuota 2010 m.} apie 16\% Londono gyventojų klausosi piratinių radijo stočių transliacijų \cite{bbc_radio_4_do_pirates_rule_the_air_waves}.

Iki 2010 m. liepos \gls{Ofcom} jau buvo sulaukęs apie 1000 privačių asmenų, legalių radijo stočių ir valstybinių tarnybų pateiktų skundų apie nelegalias transliacijas \cite{bbc_radio_4_do_pirates_rule_the_air_waves}.


\subsubsection{Techninė veiklos specifika}

Jungtinėse Valstijose leidžiamo žurnalo „Vice“ parengtame reportaže apie Londono piratines radijo stotis „London Pirate Sequences“ \cite{vice_london_pirate_sequences} laidos vedėjas Matt Mason demonstruoja įdomų šiuolaikinių Londono nelegalių transliacijų veiklos metodą. Laidos vedėjas Matt Mason, lipdamas ant vieno iš šešiolikos aukštų daugiabučių stogų, pasakoja, kad „piratai“ (nelegalių radijo transliacijų organizatoriai) paprastai įsilaužia ant tokių daugiabučių stogų ir ten įrengia savo radijo antenas, kad transliacijos aprėptų kuo didesnį plotą aplinkui \cite{vice_london_pirate_sequences}.

Peter Davies\footnote{Director of Radio Content and Broadcast Licensing, \gls{Ofcom}} teigia, kad piratų pastatytos antenos transliuoja teritorijoje, siekiančioje nuo 2 iki 10 mylių (nuo ~3,2 km iki 16 km) \cite{bbc_radio_4_do_pirates_rule_the_air_waves}.

Antenos ant stogų ypatingos tuo, kad nėra laidu tiesiogiai sujungtos su transliavimo šaltiniu (radijo studija). Vietoje to, antena didelio galingumo radijo bangomis pertransliuoja tai, ką gauna mažo galingumo infraraudonųjų spindulių (angl. \emph{infrared}) arba mikrobangų (\emph{microwave}) bangomis. Šitokiu būdu patys „piratai“ gali būti įsirengę savo studiją kur nors aplink pagrindinę anteną (iki dviejų mylių – apie ~3,2 km – atstumu nuo antenos), o pagrindinę anteną suradusiems policijos pareigūnams užtrunka išsiaiškinti, iš kur būtent yra transliuojama, kad šie galėtų uždaryti pačią studiją \cite{vice_london_pirate_sequences}.

Nors aprašytas metodas atsieti transliuotoją nuo antenos ir apsunkina policijos darbą, bet „piratai“ nelieka nesusekami, todėl šie taip pat retkačiais transliacijoms pasitelkia internetą gana nestandartiniu būdu: radijo stotis transliuojama internetu, o pati radijo bangų antena yra prijungta prie šios interneto transliacijos („klausosi“ jos kaip ir bet kuris kitas internautas). Antena radijo bangomis ištransliuoja interneto ryšiu gautą turinį, o „piratai“ (net jei ir būtų susekti) jau turi papildomą argumentą – esą jie jokių antenų nestatė, nieko apie jas nežino, o transliacijas vykdo legaliai ir tik internetu \cite{new_scientist_how_airwave_pirates_are_going_online}.

Papildomas būdas, kuriuo „piratai“ saugosi nuo policijos pareigūnų, yra toks, kad transliuojančioje radijo stotyje klausomasi savo pačių transliacijų. Išgirdę specifinį triukšmą, kuris pasigirsta policijai atjunginėjant anteną, „piratai“ sužino, kad antena buvo surasta ir kad metas trauktis iš studijos \cite{vice_london_pirate_sequences}.

„Vice“ reportaže kalbinamas „Dan“\footnote{Greičiausiai tai vardas arba pravardė.} iš nelegalios Londono radijo stoties „Flex FM“\footnote{\url{http://www.flexfm.co.uk/}} taip pat pasakoja, kad „piratai“ privalo būti organizuoti, jeigu radijo stotis nori išgyventi ilgiau negu kelias savaites; pavyzdžiui, patariama visuomet turėti atsarginę anteną, kurią būtų galima iškart įrengti po to, kai policija išjungs einamąją \cite{vice_london_pirate_sequences}.


\section{„Mikroradijas“ JAV}

Nors Jungtinėse Valstijose yra ir „įprastų“ komercinių arba mėgėjiškų piratinių radijo stočių, įdomesnis ir vien šiam regionui priskirtinas nelegalių transliacijų tipas yra vadinamieji „mikroradijai“. „Mikroradijas“ (angl. \emph{micro radio}) yra Jungtinėse Amerikos Valstijose (JAV) transliuojamos nelicencijuotos mažo galingumo radijo stotys, kuriomis skleidžiama „alternatyvi žinutė“ (angl. \emph{alternative message}) \cite[p.~134]{hc_key_concepts_in_radio_studies} oficialiosioms transliacijoms. „Mikroradijai“ Jungtinėse Valstijose (skirtingai nei Didžiojoje Britanijoje) yra nekomercinio pobūdžio bei transliuoja specifinių politinių ar religinių pažiūrų turinį.

„Mikroradijai“ yra transliuojami iš mažo galingumo siųstuvų (100 vatų, palyginus su 1000--10000 vatų siųstuvais, naudojamais „oficialių“, komercinių radijo stočių), kuriuos galima sąlyginai nebrangiai nusipirkti arba pasigaminti pačiam.

Silpnų siųstuvų, vos gyvenamųjų namų bloką arba rajoną aprėpiantis radijas JAV išpopuliarėjo devintajame dešimtmetyje \cite[p.~2]{tmc_fcc_enforcement_difficulties_with_unlicensed_micro_radio}. Populiarumas siejamas su tuo, kad komercinės radijo stotys pradėjo jungtis į konglomeratus, radijo stočių savininkų sklaida sumažėjo, kartu su tuo – ir transliuojamų nuomonių įvairovė.

Nelegalios (nelicencijuotos) transliacijos Jungtinėse Valstijose, žinoma, vyko ir prieš „mikroradijo“ erą, bet patys pasigaminti siųstuvus ir transliuoti taip, kad būtų sukeliama kuo mažesnė interferencija su legaliomis radijo stotimis (t.y., kad naujasis piratinis radijas būtų nors kiek girdimas), sugebėdavo tik elektroniką ar inžineriją studijavę asmenys. Vis dėlto, elektronikai ir inžinieriai į radijo stoties kūrimą žiūrėjo labiau kaip į technologinį, o ne kultūrinį uždavinį, t.y. kūrėjams buvo įdomiau kažką transliuoti apskritai, o ne transliuoti originalų, įdomų turinį, taigi piratinės radijo stotys JAV iki „mikroradijų eros“ transliavo tą pačią popmuziką, kurią buvo galima išgirsti ir legalių komercinių radijo stočių eteryje – „piratai“ nepasiūlė jokios alternatyvos. „Mikroradijų“ išpopuliarėjimas siejamas su informacijos apie siųstuvų gamybą sklaida – instrukcijas, kaip pačiam galima pasigaminti radijo siųstuvą, pradėjo spausdinti populiarūs žurnalai; medžiagas ir kabelius, reikalingus antenoms, buvo galima nusipirkti paprastose elektronikos parduotuvėse \cite[p.~37]{ay_low_power_fm_transmitters_electronics_now}.

Piratinės radijo stotys Britanijoje ir JAV skyrėsi tuo, kad pirmosios (britiškosios) transliacijas vykdė vedinos daugiausia komercinių interesų (popmuzikos ir reklamos transliacijų), o JAV analogės egzistavo vardan „audio pasipriešinimo“ \cite[p.~135]{hc_key_concepts_in_radio_studies}.

Nors „mikroradijus“ vadinti piratinėmis radijo stotimis tinka pagal šiame darbe vartojamą piratinių radijų apibrėžimą, T.M. Coopman išskiria šiuos skirtumus tarp „mikroradijo“ ir istoriškai vartojamos piratinio radijo sąvokos \cite[p.~3]{tmc_fcc_enforcement_difficulties_with_unlicensed_micro_radio}:

\begin{itemize}
	\item „Tradicinės“ komercinės piratinės radijo stotys dažnai keičia transliacijoms naudojamą radijo bangų dažnį ir transliacijų laiką tam, kad išvengtų susekimo, o „mikroradijai“ paprastai transliuoja fiksuotu dažniu ir turi transliacijų programą.
	\item „Tradicinės“ komercinės piratinės radijo stotys vengia susidūrimų su „valdžia“ (policija, transliacijų reguliavimo įstaigomis), o „mikroradijai“ veiklą vykdo viešai, nesislapsto, reguliavimo įstaigoms priešinasi atvirai.
	\item „Tradicinės“ komercinės piratinės radijo stotys nesiekia atgalinio ryšio iš klausytojų, veikia ganėtinai vienpusiškai; „mikroradijai“ savo ruožtu stengiasi su klausytojais užmegzti dialogą, gauti jų palaikymą.
\end{itemize}

\gls{FCC} mano, kad Jungtinėse Valstijose gali veikti iki 1000 tokių radijo stočių \cite[p.~1]{tmc_fcc_enforcement_difficulties_with_unlicensed_micro_radio}.


\subsection{„Black Liberation Radio“ (JAV)}

Ted M. Coopman savo tyrime „Free Radio v. the \gls{FCC}: A Case Study of Micro Broadcasting“ \cite{tmc_free_radio_vs_the_fcc_a_case_study_of_micro_broadcasting} pasakoja apie, jo manymu, pirmąjį žinomą „mikroradijo“ atvejį:

\begin{quotation}
	Modernus mikro--transliacijų judėjimas prasidėjo 1986 m. lapkričio 25 d. socialinių būstų \footnote{Būstų, kurie priklauso savivaldybei ir kuriuose apgyvendinti nepasiturintieji.} kvartale Springfilde, Ilinojaus valstijoje. Vieno vato stiprumo radijo stotis, pasivadinusi „Juodojo išsilaisvinimo radiju“ \footnote{angl. \emph{Black Liberation Radio}}, transliavo 107.1 FM dažniu ir buvo įkurta už maždaug 600 JAV dolerių.
	Aklas trisdešimtmetis afroamerikietis radijo stoties operatorius Mbanna Kantako pradėjo savo transliacijas, nes manė, kad afroamerikiečių visuomenė nėra užtektinai atstovaujama Springfildo vietinės žiniasklaidos. M. Kantako manė, kad radijas yra geriausias būdas pasiekti afroamerikiečių visuomenę, nes ši turėjo menką raštingumo lygį.
	Jis nurodė: „Atsižvelgiant į šiandienos technologiją, naudoti spaudą būtų tolygu raitojo pašto naudojimui vietoje pervežimų lėktuvu“.
	M. Kantako radijo stotis pasiekia apie 1000 klausytojų jo gyvenamųjų namų kvartale, transliacijas galima girdėti maždaug pusantros mylios (apie 2,4 km) aplinkui. Nors radijo stotį „puldinėjo“ policija, \gls{FCC} joje surengė reidą, ir „užkrovė“ 750 JAV dolerių baudą, stotis liko eteryje.
\end{quotation}

„Black Liberation Radio“ sėkmė paskatino kitus, ir įvairiose JAV valstijose pradėjo kurtis nauji „mikroradijai“, siekę geriau atstovauti miestams, miesteliams, rajonams ar kvartalams, kuriuose gyveno \cite[p.~135]{hc_key_concepts_in_radio_studies}.

H. Chignell vardina, kad „alternatyvių“ nelegaliųjų transliuotojų grupių buvo įvairiausių: universitetų studentai, aktyvūs krikščionys, motociklų mėgėjai („baikeriai“), abi pagrindines JAV politines partijas (demokratus ir respublikonus) palaikantieji, išeiviai iš Graikijos, hiphopo fanai ir kiti \cite[p.~136]{hc_key_concepts_in_radio_studies}.


\subsection{„Free Radio Berkeley“ (JAV)}

Kitas ankstyvasis ir garsus „mikroradijas“ buvo „Free Radio Berkeley“, transliuotas nuo 1993 m. asmens vardu Stephen Dunifer ir šiaip ne taip \gls{FCC} priverstinai uždarytas 1998 m. \cite[p.~6]{tmc_fcc_enforcement_difficulties_with_unlicensed_micro_radio}

Radijo stotis transliuota 104.1 FM dažniu, paprastai – vakarais. S. Dunifer „Free Radio Berkeley“ transliacijas pradėjo iš savo „Volvo“ automobilio, taip pat retkarčiais transliuodavo tiesiog sėdėdamas lauke (pavyzdžiui, pasilypėjęs ant kalvos ar kalno) – į patalpas radijo stotis persikėlė tik už metų, 1994 m. Ilgametis aktyvistas savo radijo stotyje grodavo muziką ir komentuodavo politinius įvykius.

Dunifer niekada nebuvo pagautas su „įkalčiais“ (betransliuodamas), bet už savo radijo stotį buvo kelis kartus baustas 20 tūkst. JAV dolerių siekiančia bauda.

Radijo stoties įkūrėjas pats pasigamino transliacijoms reikalingą įrangą, taip pat parengė viešas instrukcijas (su vaizdinėmis priemonėmis) kitiems, kurie taip pat norėtų pradėti transliuoti. Šiuo metu Dunifer prižiūri tinklalapį internete \url{www.freeradio.org}, kuriame publikuoja instrukcijas, brėžinius ir kitokią medžiagą apie transliavimo įrangos gaminimą. Dunifer taip pat organizuoja mokymus, į kuriuos susirinkę mėgėjai pasigamina savo pačių antenas.


\section{Piratinis radijas Izraelyje}

Nuo 1973 m. iki praėjusio amžiaus pabaigos prie Izraelio krantų transliavo mažiausiai 8 nelegalios radijo ir televizijos stotys \cite{soundscapes_the_worlds_last_offshore_radio_stations}. Dauguma Izraelio ofšorinių radijo stočių, skirtingai nei jų analogai prie Didžiosios Britanijos krantų, saugumo sumetimais neišplaukdavo iš pačio Izraelio teritorinių krantų.

Anot šaltinio \cite{offshore_radio_israel}, piratinės Izraelio radijo stotys taip elgėsi dėl „nerašyto susitarimo“ tarp pačių stočių ir vyriausybės įstaigų (Gynybos ir Komunikacijos ministerijų) – valstybės karinės jūrų pajėgos privalėjo žinoti, kur yra kiekvienas Izraelio laivas ir kuo jis užsiima tam, kad lengviau identifikuotų priešiškus laivus. Kadangi tarptautiniuose vandenyse transliuojančios radijo stotys būtų buvusi didesnė problema (nes būtų buvę nebeaišku, ar kuris nors konkretus laivas vykdo ne itin reikšmingą nelegalių transliacijų nusižengimą, ar, pavyzdžiui, transportuoja ginklus), Izraelio „piratams“ buvo „leista“ vykdyti nelicencijuotas transliacijas iš nekintančių pozicijų pačio Izraelio teritorijoje.

Nors beveik visose valstybėse buvo (ar yra) nelegalių transliuotojų, būtent Izraelis pasirinktas dėl ilgai gyvavusios ir unikalią įtaką padariusios radijo stoties „Voice of Peace“.

\subsection{„Voice of Peace“ (Izraelis)}

Piratinis radijas „Voice of Peace“ (angl. \emph{„taikos balsas“}) transliacijas pradėjo 1973 m. iš Viduržemio jūros. Radijo stoties įkūrėjas – Abie Nathan, buvęs avialinijų pilotas, filantropas ir humanitaras. Transliacijos truko 20 metų – iki 1993 m., kuomet laivą „MV Peace“, naudotą transliacijoms, specialiai nuskandino pats radijo stoties įkūrėjas \cite{soundscapes_voice_of_peace} (laivas buvo nuskandintas iš finansinių sumetimų, nes buvo toks senas, kad jo niekam parduoti nebeapsimokėjo, o supjaustyti dalimis metalo laužui būtų buvę per brangu \cite{soundscapes_the_worlds_last_offshore_radio_stations}).

Radijo stotis „Voice of Peace“ buvo įkurta su vieno iš „The Beatles“ dainininko John Lennon finansine pagalba \cite{soundscapes_talking_peace_in_new_york}. Radijo stoties įkūrėjas savo stotį naudojo „taikai Viduržemio jūros regione skleisti“ \cite{soundscapes_abie_nathans_peace_programme}. Radijo stoties eteryje, panašiai kaip ir pirmosiose piratinėse radijo stotyse prie Britanijos krantų, grota populiarioji muzika, transliuotos komercinės reklamos. Vis dėlto, radijo stotis neužmiršo savo pavadinimo ir transliuodavo politinio ir pacifistinio pobūdžio turinį: radijo stotis turėjo žinių programą, jos eteryje buvo aptariamos regiono politinės ir karinės aktualijos (paprastai – Izraelio karai su aplinkinėmis valstybėmis), telefoninio ryšio pagalba rengiamos diskusijų laidos.

Taip pat „Voice of Peace“ skyrėsi nuo analogių Didžiojoje Britanijoje ir apskritai šiaurės Europoje tuo, kad priešingai nei, pavyzdžiui, Britanijoje, vyriausybė pernelyg nepriekaištavo tokio „pacifistinio“ (nors ir nelegalaus) radijo egzistavimui, taigi ši radijo stotis eteryje išsilaikė net 20 metų \cite[p.~159]{os_the_noble_pirate_the_voice_of_peace_offshore_radio_station}.

„Voice of Peace“ transliavo anglų kalba. Skirtingai nei vėliau gyvavusios piratinės radijo stotys tame pačiame regione, „Voice of Peace“ transliavo iš tarptautinių vandenų (išplaukusi iš už Izraelio teritorijos krantų).

„Voice of Peace“ buvo pirmasis Izraelio komercinis radijas. Nors šiuo metu pačiame Izraelyje jau atsirado ir daugiau komercinių radijo stočių, Yaron Katz 2007 m. darbe rašo, kad šioje šalyje vis dar yra apie 100 piratinių radijo stočių, kurių dauguma transliuoja religinio pobūdžio turinį (taip pat yra ir politinių radijo stočių ir stočių, skirtų atskiroms gyventojų grupėms, pvz. arabams). Kai kurioms iš klausytojų segmentų nėra legalių analogų, kuriuos šie segmentai galėtų klausyti \cite[p.389]{yk_the_other_media_alternative_communications_in_israel}.


\section{Slaptasis radijas}

Labai panašus į piratinį radiją yra vadinamasis \emph{slaptasis radijas} (angl. \emph{clandestine radio}).

Slaptasis radijas yra nelegalių radijo transliacijų (piratinio radijo) porūšis, nes yra transliuojamas taip pat nelegaliai, bet nuo komercinio piratinio radijo skiriasi tuo, kad šio transliuotojai turi konkrečių politinių motyvų – slaptuoju radijumi transliuojama propaganda, palaikomos įvairios politinės grupuotės \cite[p.~334]{chs_encyclopedia_of_radio}. Slaptieji radijai paprastai palaiko sukilimus, skatina pilietinį karą, revoliucijas, pasipriešinimus. Taip pat ši radijo rūšis pasižymi tuo, kad transliuotojai dažnai bando apsimesti ne tuo, kuo esą, prisidengti kitos grupuotės vardu.

Slaptasis radijas yra dviejų pusių (valstybių), įsivėlusių į tarpusavio „karštąjį“ ar „šaltąjį“ karą, karinė priemonė. Taip pat slaptasis radijas dažnai (ypač karinių konfliktų metu) būna tarptautinio radijo porūšis, nes transliacijos organizuojamos ir transliuojamos iš vienos suverenios valstybės į kitą \cite[p.~749]{chs_encyclopedia_of_radio}, nors „slaptojo radijo“ transliuotojai ir neatskleidžia tikrojo skleidžiamos informacijos kūrėjo.

„Radijo enciklopedija“ skiria tris propagandinio radijo rūšis, iš kurių dvi paskutinės (antroji ir trečioji) yra laikomos slaptuoju radiju:

\begin{itemize}
	\item \textbf{„Baltosios“ radijo stotys} neslepia savo transliacijos vietos ir tikslo. Tokių radijų pavyzdžiai – „Laisvoji Europa“ ir „Amerikos balsas“. Ši rūšis nėra laikoma slaptuoju radiju.
	\item \textbf{„Pilkosios“ radijo stotys} yra transliuojamos konflikto zonoje esančių disidentų (sukilėlių), bet palaikomos užsienio šalių. Ši rūšis yra laikoma slaptuoju radiju.
	\item \textbf{„Tamsiosios“ radijo stotys} slepia transliacijos vietą, tikslus ir patį transliuotoją. Šias radijas transliuoja priešiškos užsienio valstybės. Ši rūšis yra laikoma slaptuoju radiju.
\end{itemize}

Vieni pirmųjų „slaptojo radijo“ pavyzdžių – 1926 m. sovietinės Rusijos transliacijos į Rumunijos teritoriją \cite[p.~1114]{chs_encyclopedia_of_radio}, skirtos palaikyti sovietų poziciją konflikto dėl Besarabijos metu; taip pat – Japonijos imperijos transliacijos į Mandžiūriją, pradėtos Japonijai okupavus šią teritoriją.

\subsection{„Deutscher Freiheitssender“ (Didžioji Britanija – Vokietija)}

Slaptojo radijo metodika buvo itin aktyviai naudojama Antrojo pasaulinio karo metais. Dažnas „slaptojo radijo“ naudotojas karo metu buvo Didžioji Britanija. Britai apsimetinėjo vokiška radijo stotimi (\emph{„Deutscher Freiheitssender“}, vok. „Vokietijos laisvės radijas“) ir transliavo į hitlerinę Vokietiją bei jos okupuotas teritorijas. Pirmoji tokia transliacija įvyko 1940 m. Slaptuosius radijus Antrojo pasaulinio karo metais taip pat transliuodavo Jungtinės Amerikos Valstijos (neva vokiškas „Radijas 1212“, nuo 1944 m.) \cite[p.~1115]{chs_encyclopedia_of_radio}

Viena iš sąjungininkų Antrojo pasaulinio karo metais naudotų technikų, susijusių su slaptojo radijo transliacijomis, buvo vadinamasis „ghost-voicing“ (angl. \emph{ghost} – vaiduoklis, \emph{voicing} – įgarsinimas) \cite[p.~1115]{chs_encyclopedia_of_radio}. Sąjungininkai (britai arba sovietai) stipresnių radijo siųstuvų pagalba slapta „pramušdavo“ oficialių hitlerinės Vokietijos radijo programų dažnius ir „ant viršaus“ transliuodavo kiek modifikuotą savo programą, šitaip apsimesdami oficialiuoju radiju.

Buvo stengiamasi, kad radijo klausytojas nesuprastų, kas buvo padaryta, ir klausytų „pakeistos“ radijo stoties kaip originalios. Sąjungininkai šitaip iškraipydavo nacistinės Vokietijos vadovo A. Hitlerio kalbas, stengdamiesi jį parodyti psichiškai nestabiliu ar ištransliuoti sufabrikuotas Hitlerio kalbas, kuriomis mėginta suerzinti, supriešinti klausytojus. Taip pat „ghost-voicing“ praktikai taip pat įrašinėdavo Hitlerio ar kitų nacių vadovų pasisakymus, o vėliau (po kelių mėnesių ar metų) juos petransliuodavo tuo pačiu dažniu, taip bandydami pademonstruoti Hitlerio išsakytą melą.


\section{Piratinio radijo motyvai ir prielaidos}

\subsection{Komercinės prielaidos}

Pirmasis Didžiosios Britanijos komercinis radijas LBC pradėjo transliacijas tik 1973 m. rudenį \cite{bbc_first_commercial_radio}, taip nutraukdamas 50 metų trukmės \gls{BBC} monopoliją Britanijoje. LBC ir kiti komercinio radijo pionieriai buvo atidžiai prižiūrimi vyriausybinės įstaigos \emph{Independent Broadcasting Authority} (IBA) ir veikė pagal nepalankias taisykles: komerciniai radijai privalėjo turėti pilnavertę naujienų programą, „taikytis“ į visas amžiaus grupes ir „atspindėti vietinės bendruomenės skirtumus“ \cite[p.~13]{cf_the_radio_handbook}. Tokiomis sąlygomis radijas negalėjo atrodyti patrauklus reklamdaviams, iš kurių lėšų šios radijo stotys siekė išgyventi.

IBA taip pat nustatė sąlyginai didelius mokesčius už transliacijas, muzikos kūrinių panaudojimą, įpareigojo samdyti tam tikrą radijo muzikantų kvotą. Nelegalios (piratinės) radijo stotys nebuvo varžomos šių įpareigojimų, taigi galėjo reklamdaviams pasiūlyti konkretesnę auditoriją. Komercines radijo stotis ribojanti tvarka Didžiojoje Britanijoje buvo pakeista tik 1990 m.

Jaunimo sektorius buvo daug populiaresnis tarp reklamdavių, nes po negandų pamažu atsigaunančių Europos valstybių ekonomikos pamažu kilo, taigi jaunuoliai galėdavo išleisti daugiau pinigų laisvalaikio prekėms, intensyviau vartoti.


\subsection{Neišnaudota popmuzikos paklausa}

Pirmosios piratinės radijo stotys, veikusios iš Europos jūrų (danų „Radio Mercur“, švedų „Radio Nord“, olandų „Radio Veronica“, britų „Radio Caroline“), savo bangomis transliavo populiariąją muziką jaunajai kartai. Ypač populiarus iš žanrų buvo tuo metu (penktajame--šeštajame dešimtmečiuose) Jungtinėse Valstijose užgimęs rokenrolas.

„Radijo enciklopedija“ teigia \cite[p.~237]{chs_encyclopedia_of_radio}, kad rokenrolas itin patiko europietiškai ausiai dėl dviejų priežasčių – pirma, ši muzika buvo daug energingesnė už „senesiosios kartos“ klausomą ir nacionalinių transliuotojų bangomis grojamą, o taip pat rokenrolas atspindėjo vartojimu pagrįstą, materialinių gėrybių kupiną, atsipalaidavusį gyvenimo būdą, kurio buvo pasiilgę europiečiai.

Piratinės radijo stotys Europoje taip pat buvo pirmosios, kurios į šį žemyną atnešė iki tol tik Jungtinėse Valstijose paplitusį vadinamąjį „Top 40“ formatą \cite[p.~237]{chs_encyclopedia_of_radio}.

„London Pirate Sequences“ reportaže kalbinami įvairūs esami ir buvę šiuolaikinių piratinių radijo stočių didžėjai savo veiklą taip pat grindžia visuomenės poreikiu. Nelegalios radijo stoties „Kool FM“\footnote{\url{http://www.koollondon.com/}} vadybininkas pravarde „Eastman“ argumentuoja, kad „mums reikėjo platformos paskleisti savo muziką. <...> Tomis dienomis nė vieni negrojo tokio tipo muzikos“. „Eastman“'ui pritaria ir kalbinamas „Jammer“, Londono prodiuseris ir vakarėlių vedėjas: „Tai yra būdas išplatinti savo muziką; tai yra būdas pasiekti, kad tavo muzika būtų išgirsta. Jeigu myli muziką, visada prisiimsi šią riziką“ \cite{vice_london_pirate_sequences}.

Sarah Lockhard, neseniai (2010 m.) „legalizuotos“ piratinės radijo stoties „Rinse FM“ vadovė, laidos metu teigia, kad „oficialusis radijas visada sekė piratinį radiją“. Pasak laidos pašnekovės, oficialūs radijai visada grojo jau išpopuliarintus muzikos kūrinius ar stilius, o šitoks populiarinimas kartais gali užtrukti iki 10 metų. Pasak S. Lockhard, piratiniai radijai buvo tie, kurie išpopuliarino tai, ką dabar groja legalios radijo stotys \cite{bbc_radio_4_do_pirates_rule_the_air_waves}.

\gls{Ofcom} atstovas Peter Davies tvirtina, kad oficialus reguliuotojas norėtų ne uždaryti visas „originaliasias“ piratines radijo stotis, bet to, kad šios gautų licencijas; vis dėlto, pačios piratinės radijo stotys pernelyg nesiveržia į konkursus dėl atsilaisvinusių bangų ruožų, nes, pasak laidos pašnekovo, „tuomet jiems tenka konkuruoti su visais kitais transliuotojais“, t.y. įrodyti, kad piratai geba sukurti kokybišką ir turtingą radijo programą \cite{bbc_radio_4_do_pirates_rule_the_air_waves}.


\subsection{Darbo radijuje patirtis}

T. Nelson, „\gls{BBC} Radio 4“ radijo laidų vedėjas, pasakoja buvęs vienos iš piratinių radijo stočių „Kiss FM“ didžėjus. T. Nelson tvirtina, kad būtent patirtis piratinėje radijo stotyje jį atvedė į legalų darbą \gls{BBC} – kitaip „jie net nebūtų į mane žiūrėję“ \cite{bbc_radio_4_do_pirates_rule_the_air_waves}.

Piratinis radijas kelią į „oficialųjį“ radiją atvėrė ir tokiems žymiems radijo stočių didžėjams kaip John Peel ar Tony Blackburn.

\subsection{„Žodžio laisvės“ argumentacija}

Tony Pine, reportaže kalbinamas apie 1964 m. šalia Britanijos krantų transliavusios nelegalios (ofšorinės) radijo stoties „Radio Sutch“ („Radio City“) inžinierius \cite{tony_pine}, tvirtina, jog „mes \emph{(radijo stoties transliuotojai – LV.)} manėm, kad turėti nuosavą radijo stotį buvo žmonių teisė“. Tyrinėtojas John Anderson savo ruožtu teigia, kad piratinės radijo stotys yra savotiškas bandymas visuomenei „atsiimti“ radijo bangas, kurios jai priklauso, bet yra pasisavintos daugiausia komercinių radijo stočių \cite[p.~1-2]{ja_a_can_of_worms_pirate_radio_as_public_intransigence_of_the_public_airwaves}. Autorius rašo:

\begin{quotation}
	<...> Susidaro keista situacija: visuomenė lyg ir turėtų būti radijo bangų savininkė, bet vis dėlto ji pati neturi teisės šiomis bangomis naudotis. Vyriausybė, tarnaujanti kaip šio viešojo resurso skirstytojas per licencijavimą, šią visuomenės teisę pavertė vien į teisę klausytis, teisę gauti transliacijas, bet ne į teisę pačiam transliuoti.
\end{quotation}

Senokos (išleistos 1994 m.), bet populiarios radijo bangų siųstuvo konstravimo instrukcijos, išleistos Jungtinėse Valstijose asmens, pasivadinusio slapyvardžiu „Zeke Teflon“, įvadas \cite{zt_complete_manual_of_pirate_radio} kupinas panašių transliavimo motyvų:

\begin{quotation}
	Šis pamfletas skirtas tiems, kurie turi žinutę, kurią norėtų ištransliuoti, bet neturi būdų tai padaryti. <...> Taigi, jei susidomėjote galimybe pateikti alternatyvą korporacijų valdomiems laikraščiams ir žurnalams, žliumbiantiems religiniams radijams ir televizijoms, reakcingai košelei, transliuojamai komercinių \emph{(radijo – L.V.)} stočių, kuri sukurta taip, kad tik neįžeistų stoties reklamdavių ir savininkų, ir beginklio „gyvūnai--ir--britiškas--akcentas“ programavimo viešosiose (priklausančiose valstybei) radijo stotyse, skaitykite toliau.
\end{quotation}

Pamfletas toliau vardina tokius nelegalių („savarankiškų“) transliacijų motyvus: steigti laikraštį – labai brangu; steigti televiziją – dar brangiau; radijas – įkandama, bet radijo bangų licencijos yra per brangios, sunkiai gaunamos ir gali būti atimtos. Vėliau anarchistinių pažiūrų „Zeke“ argumentuoja, kad komunikacijų laisvė yra viena iš pagrindinių žmogaus laisvių, o „valstybė jokių laisvių savo piliečiams nedalina – jas reikia pasiimti pačiam“.


\section{Bėdos ir baudos}

\gls{Ofcom} vardina šias problemas, kurias sukelia nelegalios radijo transliacijos \cite{ofcom_pirate_radio_fines} \cite{bbc_radio_4_do_pirates_rule_the_air_waves}:

\begin{itemize}
	\item Klausytojai negali klausytis legalių radijo stočių, nes „piratai“ pasivagia joms priskirtą radijo bangų ruožą.
	\item „Piratai“, besibraudami į privačias valdas (daugiabučius, kuriuose stato savo transliacijų antenas), pridaro žalos ir vagia (pavyzdžiui, daugiabučių liftų elektrą transliavimo antenoms maitinti).
	\item „Piratai“ dėl savo veiklos nemoka jokių mokesčių valstybei, taigi vagia, o ne prisideda prie savo bendruomenių.
	\item Nelegalios transliacijos trikdo arba visiškai blokuoja gyvybiškai svarbių tarnybų – gaisrinės \cite{ofcom_illegal_broadcasting_factsheet}, greitosios medicinos pagalbos, policijos, lėktuvų eismo kontrolės – darbą.
	\item „Piratai“ kartais transliacijos antenas nuo norinčiųjų jas nuimti apsaugo pavojingais savadarbiais įrengimais, pavyzdžiui, aštriais skutimosi peiliukais arba aukštos įtampos kabeliu. \cite{ofcom_illegal_broadcasting_factsheet}
	\item „Piratai“ išnaudoja jaunus ir negarsius diskotekų vedėjus, norinčius tapti žinomais, reikalaudami iš jų piniginio atlygio už tai, kad šie patektų į piratinės radijo stoties eterį. \cite{ofcom_illegal_broadcasting_factsheet}
	\item Kai kurios piratinės radijo stotys transliuoja rasistinio arba skatinančio smurtą pobūdžio turinį. \cite{ofcom_illegal_broadcasting_factsheet}
\end{itemize}

Laidoje vienas iš \gls{Ofcom} atstovų Jim McNally nelegalius transliuotojus kaltina ir „ryšiais“ su narkotikų vartojimu bei platinimu, nelegalių šaltųjų ir šaunamųjų ginklų turėjimu, pinigų „plovimu“, nors ir teigia, kad negali pateikti tikslesnės informacijos ar skaičių apie tokius atvejus \cite{bbc_radio_4_do_pirates_rule_the_air_waves}.

Didžiojoje Britanijoje nelegalios (nelicencijuotos) transliacijos radijo bangomis laikomos kriminaliniu nusikaltimu (tai, be kitų ypatybių, reiškia ir tai, kad bet kuris asmuo, žinantis apie „piratinę“ veiklą, privalo apie ją pranešti arba taip pat gali būti baudžiamas). Pažeidėjai gali būti baudžiami neriboto dydžio pinigine bauda ir/arba dviem metais kalėjimo. Papildomai nubaustam asmeniui penkis metus draudžiama dirbti bet kurioje legalioje radijo stotyje \cite{ofcom_pirate_radio_fines}. Mažiausiai vienam asmeniui, Didžiojoje Britanijoje nubaustam už nelegalias radijo transliacijas, buvo uždrausta užlipti ant bet kurio namo stogo Londone \cite{ofcom_pirate_radio_rooftop_ban}.


\section{Piratinis radijas Lietuvoje}

Panašu, kad Lietuvoje bandyta „piratauti“ seniai ir nesėkmingai: 2004 m. vasarą „Lietuvos radijo ir televizijos komisijos iniciatyva nutraukta nelegali radijo programų transliacija 99,1 MHz dažniu Palangoje“ \cite{rtk_nelegali_transliacija_palangoje}. Transliuota iš tuo metu populiarios kavinės „Laukinių vakarų salūnas“, esančios Basanavičiaus g. \cite{ve_palangoje_aptikta_nelegali_radijo_stotis} Kavinės savininkams tuomet grėsė bauda nuo 2500 iki 5000 Lt.

Transliacijų tvarką Lietuvoje šiuo metu\footnote{2011 m. rugpjūtis.} reglamentuoja Lietuvos Respublikos Elektroninių ryšių įstatymas \cite{lr_elektroniniu_rysiu_istatymas}. Pasak įstatymo, licencijas transliacijoms radijo bangomis, bendradarbiaudama su Ryšių reguliavimo tarnyba, išduoda Lietuvos radijo ir televizijos komisija.

Baudas už nelegalias transliacijas nustato Administracinių teisės pažeidimų kodeksas \cite{lr_administraciniu_teises_pazeidimu_kodeksas}. Aktualios redakcijos \footnote{2011 m. rugpjūtis.} 152(4) straipsnyje „Radijo ryšio įrenginių ir telekomunikacijų galinių įrenginių techninio reglamento sąlygų pažeidimas“ numatytos baudos nuo dviejų tūkstančių penkių šimtų iki penkių tūkstančių litų su aparatūros arba įrenginių konfiskavimu ar be konfiskavimo. Pakartotiniai prasižengėliai baudžiami nuo penkių tūkstančių iki dešimties tūkstančių litų bauda – taip pat su galima turto konfiskacija.


\section{(Nebe)piratinis radijas internete}

Vykdyti nelegalias (piratines) transliacijas yra sudėtinga ir pavojinga veikla – privaloma išmanyti viską, nuo radijo siųstuvų veikimo bei elektroinžinerijos pagrindų iki įstatyminės bazės ir aplink siųstuvą esančios auditorijos skonio. Atrodytų, paprasčiausia išeitis būtų šiems nelegaliems radijams „išsikraustyti“ į internetą ir pradėti įstatymiškai nereguliuojamas transliacijas tiesiog ten.

Interneto įtaką vertina ir „London Pirate Sequences“ reportažo pašnekovai. Pavyzdžiui, Logan Sama, dabartinis\footnote{Reportažo filmavimo metu.} „Kiss FM“ radijo stoties\footnote{\url{http://www.totalkiss.com/}} (anksčiau buvusios piratine, bet legalizuotos 1990 m.) ir buvęs „Rinse FM“\footnote{\url{http://rinse.fm/}} (nuo 1994 m. transliavusių „piratų“, legalizuotų 2010 m.) didžėjus, internetą įvardina kaip turintį „papildomų patirties sluoksnių“ ir pareiškia, kad interneto pagalba „kiekvienas dabar gali turėti po piratinį radiją“ \cite{vice_london_pirate_sequences}.

Dalis nelegalių transliuotojų savo programas jau pertransliuoja ir internetu. Vis dėlto, nelegalios transliacijos „klasikinėmis“ radijo bangomis, net ir išpopuliarėjus internetui, neišnyko. Priežasčių reikėtų ieškoti pačių internetinių transliacijų trūkumuose \cite[p.~758]{chs_encyclopedia_of_radio}:

\begin{itemize}
	\item \textbf{Didelė konkurencija}. „Įėjimas“ į internetinių radijo transliacijų rinką yra itin paprastas, transliuoti ką nors internetu gali pradėti kiekvienas panorėjęs. Dėl šios priežasties internetinių (internetu transliuojamų) radijo stočių yra nepalyginamai daugiau nei paprastų radijo stočių, taigi naujai šios chaotiškos rinkos dalyvei (piratinei radijo stočiai) tenka kovoti su milžiniška konkurencija.
	\item \textbf{Reklamdavių stoka}. Dėl tos pačios priežasties – per didelio pasirinkimo klausytojui – internetinis radijas nėra toks patrauklus reklamdavių tarpe. Klausytojas visada gali persijungti į kitą internetinio radijo stotį, kurio eteryje reklamos nėra.
\end{itemize}

Viena iš Didžiosios Britanijos nacionalinio reguliuotojo \gls{Ofcom} regimų išeičių yra artėjanti skaitmeninio radijo (angl. DAB – \emph{„digital audio broadcasting“}) era \cite{bbc_radio_4_do_pirates_rule_the_air_waves}. Skaitmeninio radijo dėka galima transliuoti daug radijo stočių tuo pačiu bangų ruožu, ir nebereikėtų „dalintis“ radijo bangų licencijavimu \cite{worlddab}.


%
% Išvados
%

\vukfConclusion

% **Išvadose** suformuluojamos svarbiausios darbo išvados, pateikiamos rekomendacijos ir pasiūlymai, numatomos tolesnės temos tyrimo perspektyvos. Išvadose neturi būti pateikiama darbo santrauka. Privalomas išvadų sąsajumas su darbo tikslu. Rekomenduojama išvadų apimtis – 5–10 proc. kursinio darbo apimties.

Pirmieji ir reikšmingiausi piratiniai radijai Europoje įkurti praėjusio amžiaus šeštajame – septintajame dešimtmečiuose, ypač – Didžiojoje Britanijoje. Piratinės transliacijos buvo pelningas verslas, bet kartu ir pasitarnavo „atnešdamos“ į Europos radijo erdvę naujus muzikos stilius, formatus, supratimą apie radijo laidų vedėjus ir apskritai laisvesnį požiūrį į radiją.

Garsiausios šiuolaikinės piratinės radijo stotys veikia didžiuosiuose Didžiosios Britanijos miestuose. „Piratai“ gerai išmano šiuolaikinę technologiją, moka slapstytis. Britų nelegalūs transliuotojai argumentuoja esantys naujų muzikos srovių ieškotojai ir populiarintojai; juos bandanti kontroliuoti įstaiga „Ofcom“ nelegalus vertina kaip nesąžiningus bei pavojingus pažeidėjus.

Visiškai kita linkme pasukęs „mikroradijas“ Jungtinėse Amerikos Valstijose panašiais (bet atviresniais, drąsesniais) metodais kaip Britanijoje kovoja, pasak pačių transliuotojų, su žiniasklaidos priemonių centralizacija, ir transliuoja tiems, kurie gyvena arčiausiai ir kurių problemos yra geriausiai žinomos.

„Voice of Peace“ radijas Izraelyje, galbūt vienintelis, kurio valstybinės įstaigos nesistengė uždaryti, buvo gerai organizuota, 20 metų išsilaikiusi komercinė, bet nekomercinių motyvų įstaiga, skatinusi Viduržemio jūros regioną susitaikyti tarpusavyje.

Piratinių radijo stočių metodika „slaptųjų radijų“ pavadinimu naudota ir pačių valstybių, kariavusių tarpusavyje „šaltąjį“ arba „karštąjį“ karą ir mačiusių radijo bangas tik kaip dar vieną kanalą, kuriuo galima pulti ir gintis.

Piratinės radijo stotys, nors ir apipintos romantika ir tam tikra mistika, valstybėje yra laikomos nusikaltimu. Nelegalūs transliuotojai teigia, kad radijo bangomis platina laisvą žodį ir naujausią muziką, bet iš tikrųjų trikdo svarbių (kartais – ir gyvybiškai) tarnybų darbą bei nesąžiningai elgiasi su įmonėmis, kurios naudojamą radijo bangų ruožą nusipirko už dideles lėšas.

Interneto erdvė yra puiki alternatyva mėgėjiškoms, savanoriškoms transliacijoms vykdyti, bet nelegalūs transliuotojai radijo bangomis ten „kraustytis“ neskuba, nes tai jiems yra komerciškai nenaudinga, o konkurencija per didelė.


%
% Bibliografija
%

\bibliography{kursinis-piratines-radijo-stotys-bibliografija}

\end{document}
